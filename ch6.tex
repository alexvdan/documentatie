\chapter{Implementation Details}
% \chapter{Detalii de implementare}
\label{cap:implementare}


Ponderea acestui capitol relativ la întreaga lucrare este de 20-30\%.

Conține detalii de implementare: 
\begin{itemize}
  \item organizarea codului sursă, organizarea logică a codului (module, ierarhii de clase)
  \item descrierea claselor, funcțiilor, API-urilor importante ale aplicației
  \item descrierea la nivel de implementare a algoritmilor principali
  \item descrierea părților mai dificile
  \item alte detalii de implementare relevante, specifice fiecărei aplicații
\end{itemize}

Descrierea implementării trebuie să reflecte modul în care ea corespunde (se mapează) design-ului. 

Nu se vor da detalii irelevante. Descrierea codului trebuie gândită ca un ghid de parcurgere a codului sursă de către cineva care vrea să continue proiectul vostru. 

Exemplu de cod:
\lstset{language=C,frame=single, showstringspaces=false}
\begin{lstlisting}
# include <stdio.h>
  
int main (int argc, char **argv)
{
  int i;
    
  for (i=0; i<argc; i++)
    printf("argv[%d] = %s\n", i, argv[i]);
    
  return 0;
}
\end{lstlisting}
