
%\chapter{Conclusions}
 \chapter{Conluzii}
\label{cap:concluzii}
\section{Privire de ansamblu asupra sistemului}

În urma eforturilor depuse, s-a realizat un sistem de prevenirea a intruziunilor care îndeplinește aproape în întregime obiectivele inițial propuse, mici inconveniente fiind la partea de performanță în detecție a sistemului. În cea ce privește performanța sistemului, în cazul detecției atacurilor de SQL injection(precum se poate vedea și în capitolul 7) procentul de fals pozitiv este mult mai mare decât se intenționa inițial(12.66\% în loc de 3-4\%), însă acest lucru datorându-se în principal setului mic de date de antrenare avute la dispoziție. 

Prin implementarea detecției atacurilor SQL injection folosind tehnici de machine learning, s-au dobândit cunoștințe valoroase și experiență ce pot fi folosite în viitoare proiecte. Întrucât în practică, abordarea acestui subiect a reprezentat un lucru nou, o mare parte din timpul investit în dezvoltarea sistemului a reprezentat documentarea și acumularea de noi cunoștințe și experiență pentru rezolvarea unor probleme cu tehnici de machine learning. 

\section{Dezvoltări ulterioare}

Datorită unei abordări modulare a dezvoltării sistemului, acestuia îi pot fi adăugate cu ușurință noi funcționalități. 

Sistemul prezintă două posibilități majore de dezvoltări ulterioare, acestea fiind aferente celor două tipuri de protecție implementată. În cazul protecției împotrivă adreselor IP malițioase, lista folosită momentan se poate extinde cu ușurință sau prin mici modificări în cod se pot adaugă noi liste. 

Protecția bazată pe detecția unui anumit conținut în URL poate fi cu ușurință extinsă, prin adăugarea de noi model de machine learning sau noi logici de detecție, ambele cazuri necesitând mici modificări în codul sursă. 

%Cuprinde:
%
%\begin{itemize}
% \item un rezumat al contribuțiilor aduse: ce s-a realizat, relativ la ce s-a propus, în ce constă experiența acumulată, care au fost punctele dificile atinse și rezolvată, recomandări pentru alții care abordează tema, la ce este bun ce s-a obținut etc.
% 
% \item a analiză critică a rezultatelor obținute: avantaje, dezavantaje, limitări
% 
% \item o descriere a posibilelor dezvoltări și îmbunătățiri ulterioare
%\end{itemize}
%
%Poate fi organizat pe secțiuni, dacă se dorește.
%
%Se întinde pe aproximativ 1-2 pagini. 






