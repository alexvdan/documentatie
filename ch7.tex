
\chapter{Tests and Results}
% \chapter{Teste și rezultate experimentale}
\label{cap:rezultate}

Ponderea acestui capitol relativ la întreaga lucrare este de 5-10\%.

Aici sunt prezentate metodele de validare a soluțiilor/sistemului descris în capitolele anterioare, scenariile de testare a corectitudinii funcționale, a utilizabilității, performanței etc.   

Rezultatele testelor experimentale necesită, în general interpretări (dacă rezultatele obținute corespund așteptărilor, intuițiilor cititorului, de ce apar variații/excepții etc.) și comparații cu rezultatele altor metode similare. 

Sistemele de testare și testele propriu-zise trebuie descrise detaliat astfel încât să poată fi reproduse și de alții care poate vor să-și compare soluțiile lor cu a voastră (eventual, codul testelor poate fi pus în anexe). Dacă se poate alegeți pentru evaluarea sistemului vostru benchmark-uri (pachete de testare) dedicate, astfel încât comparația cu alte sisteme să poată fi făcută mai ușor. În plus, astfel de teste sunt mult mai complete și mai realiste decât cele dezvoltate de voi. Oricum, încercați ca testele efectuate să nu fie triviale, ci să acopere scenarii cât mai reale, mai complexe și mai relevante ale funcționării sistemului vostru. 

\section{Functional Tests}
% \section{Teste de funcționalitate}



\section{Performance Tests}
% \section{Teste de performanță}



