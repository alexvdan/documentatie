%%%%%%%%%%%%%%%%%%%%%%%%%%%%%%%%%%%%%%%%%%%%%%%%%%%%%%%%%%%%%%%%%%%%%%%%%%%%%
%%%
%%% File: thesis.tex, version 0.1, May 2010
%%%
%%% =============================================
%%% This file contains a template that can be used with the package
%%% cs.sty and LaTeX2e to produce a thesis that meets the requirements
%%% of the Computer Science Department from the Technical University of Cluj-Napoca
%%%%%%%%%%%%%%%%%%%%%%%%%%%%%%%%%%%%%%%%%%%%%%%%%%%%%%%%%%%%%%%%%%%%%%%%%%%%%

\documentclass[12pt,a4paper,twoside,openright]{report}

\usepackage{cs}

% !!!!!!!!!!!!!!!!!!!!!!!!!!!! IMPORTANT NOTE !!!!!!!!!!!!!!!!!!!!!!!
% FOR THESIS IN ROMANIAN LANGUAGE COMMENT OUT THE FOLLOWING LINE !!!!
\usepackage[romanian]{babel}


\graphicspath{{figures/}}
\ifpdf
  \DeclareGraphicsExtensions{.pdf,.jpeg,.png}
\else
  \DeclareGraphicsExtensions{.eps}
\fi

% \mastersthesis
\diplomathesis
% \leftchapter
% \centerchapter
% \rightchapter
\singlespace
% \oneandhalfspace
% \doublespace

\renewcommand{\thesisauthor}{Alexandru Daniel Vid}    %% Your name.
\renewcommand{\thesismonth}{Februarie}     %% Your month of graduation.
\renewcommand{\thesisyear}{2019}      %% Your year of graduation.
\renewcommand{\thesistitle}{Implementarea unui reverse proxy pentru prevenirea atacurilor la nivel de reţea}

\renewcommand{\thesissupervisorname}{ Ș.l. Dr. Ing. Ciprian Pavel Oprișa}


%\renewcommand{\thesisdedication}{To my beloved wife and parents}

\begin{document}


% ================================================================================
% ======================== FIRST TITLE PAGE =====================================
% ================================================================================

\begin{titlepage}

\thispagestyle{firststylewithoutfooter}

% \fancyhead{}
% \fancyhead[C]{\includegraphics[width=20pt]{header-utcn-engleza.png}}
% \chead{\includegraphics[width=20pt]{header-utcn-engleza.png}}


\begin{center}
% {\scshape \universitynameenglish} \\
% {\scshape \facultynameenglish} \\
% {\scshape \departmentnameenglish} \\

% {\scshape \universitynameromanian} \\
{\scshape \facultynameromanian} \\
{\scshape \departmentnameromanian} \\


\vspace{6cm}

\thesistitlesize {\textbf{\thesistitle}\\}
\vspace {1cm}

\thesistypesize \textbf{\thesistyperomanian}\\
% \Large \textbf{\thesistyperomanian}\\


\vspace{2cm}

% \thesisauthortypesize \thesisauthortypeenglish \\ \textbf{\thesisauthor} \\
\thesisauthortypesize \thesisauthortyperomanian \\ \textbf{\thesisauthor} \\

\vspace{1cm}

% \thesissupervisorsize \thesissupervisorenglish \\ \textbf{\thesissupervisorname}\\
\thesissupervisorsize \thesissupervisorromanian \\ \textbf{\thesissupervisorname}\\



\vspace{\stretch{1}}
{\thesismonth} {\thesisyear} \\
\end{center}
\end{titlepage}

\begin{titlepage}
\phantom{1}
\end{titlepage}


% ================================================================================
% ======================== SECOND TITLE PAGE =====================================
% ================================================================================

\begin{titlepage}

\begin{center}

\thispagestyle{firststylewithfooter}

% {\scshape \universitynameenglish} \\
% {\scshape \facultynameenglish} \\
% {\scshape \departmentnameenglish} \\

% {\scshape \universitynameromanian} \\
{\scshape \facultynameromanian} \\
{\scshape \departmentnameromanian} \\

\vspace{1cm}

\newcolumntype{R}{>{\raggedleft\arraybackslash}X}%
\begin{tabularx}{\textwidth}{lR}
% {\scshape \facultydeanenglish} & {\scshape \deptmanagerenglish} \\
{\scshape \facultydeanromanian} & {\scshape \deptmanagerromanian} \\
\facultydeanname & \deptmanagername\\
\end{tabularx}

\vspace {2cm}

\thesistitlesize {\textbf{\thesistitle}\\}
\vspace {1cm}

% \thesistypesize \textbf{\thesistypeenglish}\\
\Large \textbf{\thesistyperomanian}\\

\vspace{1cm}

\end{center}

% \vspace{1cm}

\begin{flushleft}
\begin{enumerate}
%   \item \textbf{\thesisauthortypeenglish}: \thesisauthor
  \item \thesisauthortyperomanian: \thesisauthor
 
%  \item \textbf{\thesissupervisorenglish}: \thesissupervisorname
 \item \thesissupervisorromanian: \thesissupervisorname
 
%  \item \textbf{\thesiscontentsenglish}: Thesis presentation, suprvisior evaluation, chapter 1, chapter 2, \dots, chapter n, References, Anexes, CD.
 \item \thesiscontentsromanian: Pagina de prezentare, cuprins, listă de figuri, introducere, obiective și specificații, studiu bibliografic, fundamente teoretice, analiză și proiectare, detalii de implementare, teste și rezultate experimentale, manual utilizator, concluzii, bibliografie, anexe, CD.
 
%  \item \textbf{\thesisworkingplaceenglish}: UTCN, Cluj-Napoca
 \item \thesisworkingplaceromanian: UTCN, Cluj-Napoca

%  \item \textbf{\thesisadvisorsenglish}: Donald Knuth, Leslie Lamport, others \dots
 \item \thesisadvisorsromanian: \thesissupervisorname 

%  \item \textbf{\thesisbegindateenglish}: \dotfill
 \item \thesisbegindateromanian: 05.05.2018

%  \item \textbf{\thesisenddateenglish}: \dotfill
 \item \thesisenddateromanian: 18.02.2019

\end{enumerate}

\end{flushleft}

\vspace{0.5cm}

\begin{center}

\newcolumntype{R}{>{\raggedleft\arraybackslash}X}%
\begin{tabularx}{\textwidth}{lR}
% {\thesissignatureenglish} {\thesissupervisorenglish} & {\thesissignatureenglish} {\thesisauthortypeenglish} \\
{\thesissignatureromanian} {\thesissupervisorromanian} & {\thesissignatureromanian} {\thesisauthortyperomanian} \\
\thesissupervisorname & \thesisauthor \\
\end{tabularx}

\vspace{\stretch{1}}
{\thesismonth} {\thesisyear} \\

\end{center}

\end{titlepage}


\begin{titlepage}
\phantom{1}
\end{titlepage}


% ================================================================================
% ======================== THIRD TITLE PAGE =====================================
% ================================================================================

\begin{titlepage}

\begin{center}
\thispagestyle{firststylewithfooter}

% {\scshape \universitynameenglish} \\
% {\scshape \facultynameenglish} \\
% {\scshape \departmentnameenglish} \\

% {\scshape \universitynameromanian} \\
{\scshape \facultynameromanian} \\
{\scshape \departmentnameromanian} \\
\end{center}

\vspace{3cm}

\begin{center}
% \autheticitydeclarationsize \textbf{\autheticitydeclarationenglish}
\autheticitydeclarationsize \textbf{\autheticitydeclarationromanian}
\end{center}

\vspace{1cm}

Subsemnatul \textit{\thesisauthor}, legitimat cu \textit{CI} seria \textit{XH} numărul \textit{866549}, CNP \textit{1950417055056}, autorul lucrării \textit{\thesistitle} elaborată în vederea susținerii examenului de finalizare a studiilor de masterat la Facultatea de Automatică și Calculatoare, Departamentul Calculatoare, Specializarea \textit{Calculatoare} din cadrul Universității Tehnice din Cluj-Napoca, sesiunea \textit{\thesismonth} a anului univeristar \textit{2018/2019}, declar pe proprie răspundere, că această lucrare este rezultatul propriei mele activități intelectuale, pe baza cercetărilor mele și pe baza informatiilor obținute din surse care au fost citate în textul lucrării și în bibliografie.

Declar că această lucrare nu conține porțiuni plagiate, iar sursele bibliografice au fost folosite cu respectarea legislației române și a convențiilor internaționale privind drepturile de autor.

Declar, de asemenea, că această lucrare  nu a mai fost prezentată în fața unei alte comisii de examen de licență sau disertație.

În cazul constatării ulterioare a unor declarații false, voi suporta sancțiunile administrative, respectiv, \textit{anularea examenului de licență}.


\vspace{2cm}

\begin{center}

\newcolumntype{R}{>{\raggedleft\arraybackslash}X}%
\begin{tabularx}{\textwidth}{lR}
% Cluj-Napoca & {\thesissignatureenglish{ }\thesisauthortypeenglish}\\
% date  & {\thesisauthor} \\
Cluj-Napoca & {\thesissignatureromanian}\\
12.02.2019  & {\thesisauthor} \\ 
\end{tabularx}

\end{center}


\end{titlepage}


\begin{titlepage}
\phantom{1}
\end{titlepage}


%\pagestyle{headings}

% ABSTRACT
\begin{abstract}
\textit{\thesistitle}  încapsulează trăsăturile normale ale unui reverse proxy oferind în plus protecție împotriva posibilelor atacuri la nivel de rețea. Produsul oferă protecție prin implementarea a două tipuri de prevenire a atacurilor: listă neagră pentru adresele IP și  blocarea unor request-uri pe baza URL-urilor prin analiza conținutului lor. Pentru demonstrarea fiecărei metode în parte, s-a folosit: blocarea adreselor IP utilizate de rețeaua Tor pentru lista de IP-uri "negre" și detectarea atacurilor de SQL injection pentru protecția împotriva URL-urilor cu conținut malițios. Detecția atacurilor de tipul SQL injection se realizează prin analiza URI-urilor trimise de către clienți, în relație cu un model antrenat anterior folosind machine learning. Ambele implementări asigură adăugarea cu ușurință de noi detecții, lista de ip-uri blocate fiind acesibila utilizatorului atât pentru vizualizare cât și pentru editare. 
\end{abstract}

\begin{titlepage}
\phantom{1}
\end{titlepage}



%\thesistitle                    %% Generate the title page.
%\authordeclarationpage                %% Generate the declaration page.

\pagenumbering{Roman}
\setcounter{page}{1}

\tableofcontents
\newpage

%\listoftables

\listoffigures

\begin{titlepage}
\phantom{1}
\end{titlepage}

% \clearpage
\newpage

\pagenumbering{arabic}
\setcounter{page}{1}

\pagestyle{normalpagestyle}
\renewcommand{\chaptermark}[1]{ \markboth{\thechapter. #1}{} }
\renewcommand{\sectionmark}[1]{ \markright{\thesection. #1}{} }



%\chapter{Introduction}
\chapter{Introducere}
\label{cap:Introducere}


\section{Context}

In general, tentativele da expluatarea vulnerabilitatilor unei aplicatii vin sub forma de input catre aceasta, generate de catrea un atacator care intentioneaza sa intrerupa activitatea sau sa preia controlul aplicatiei. Un sitem de prevenire a intruziunilor (IPS) are rolul de a sta intre aplicatie si clientii acesteia, si de a prevenie expluatarea unor astfel de vulnerabilitati.

Prin folosirea unu reverse proxy pentru accesarea resurselor unui server de catre clineti, poate sa aduca numeroase beneficii procesului de administrare a serverului \cite{top_8}. Spre deosebire de un forward proxy care e un intermediar puntru un o serie de clienti prestabiliti, pentru a accesa orice server, un reverse proxy e un intermediar pentru o serie de servere prestabilite pentru a fi accesate de orice client. Unul dintre avantajele folosirii unui reverse proxy este centralizarea intregului trafic al serverului/serverelor intr-un singur punct de acces, aceasta fiind si principala caracteristica expluatata de acest proiect pentru filtrarea ip-urilor nedorite(in cazul nostru cele utilizate de reteaua Tor) si verificarea URI-urilor pentru posibile atacuri de SQL injection.

\begin{figure}[h]
	\centering
	\includegraphics[width=0.5\textwidth]{reverse-proxy-02-1.jpg}
	\caption{Incapsula diagrama reverse proxy}
	\label{fig:reverse-proxy}
\end{figure}

Figura ~\ref{fig:reverse-proxy} ilustreaza modul de functionare al unui reverse proxy in relatie cu serverele aferente si posibili clienti. \\


Conform topului alcatuit de fundatia OWASP cu cele mai mari 10 riscuri ale aplicatiilor in 2017 \cite{owasp}, SQL injection e considerat a fi cea mai mare vulnerabilitate a aplicatiilor web. Acest lucru se datoreaza fatului ca mare parte din aceste aplicatii se bazeaza pe procesearea continutului furnizat de catre utilizatori. Atacurile de tipul SQL injection costau in faptul ca datele furnizate de catre utilizator sunt introduse in interogari SQL, unde acestea sunt tratate ca si cod executabil \cite{classification_and_countermeasures}. Aplicatiile web vulnerabile la sqli injection pot permite unor utilizatori neautorizati sa faca interogari intr-o baza de date asupra unor date la care nu ar avea acces in mod normal. Folosind acest tip de comportament neautorizat, un astfel de utilizator poate sa obtina accesul la informatii sensibile ale clientilor, dar si a administratorilor aplicatiei, precum credentiale sau date personale. Aceasta vulnerabilitate poate sa duca la furt de identitate sau frauda. 

In cazul retelei Tor, aceasta le permite utilizatorilor sa navigheze pe internet anonim. Anonimitatea online este importanta insa in multe cazuri aplicatiile web trebuie sa stie cine se conecteaza la aceasta pentru a le putea determina intentiile. Numele de Tor vine de la "the onion router" care sugereaza modul de operare al retelei. Fiecare participant la retea devine un nod de transfer, iar traficul retelei traverseaza o serie de astfel noduri pana sa ajunga la nodul de iesire ce creaza conexiunea cu destinatia dorita. Pachetele sunt criptate in mai multe "straturi", fiecare nod decriptand un singur strat de unde poate afla doar destinatia nodului urmator. Cand pachetul ajunge la ultimul nod, acesta trimite continutul la destinatie fara sa dezvalui identitatea sursei. Aceasta anonimitate usearaza desfasurarea atacurilor online. Conform datelor din reteaua organizatiei CloudFlare 94\% din traficul provenit din reteaua Tor este alcatuit din request-uri automate de origine malitioasa \cite{tor_trouble}.


%\section{Motivation}
 \section{Motivație}
In piata actuala exista multe sitem de prevenire a intruziunilor ce ofera atat caracteristicile unui reverse proxy, cat si cele de securitate. Aceste caracteristici sunt oferite fie ca si produse individualea, fie ca si produse ce le incorporeaza pe ambele. Cu toate acestea, produsele de acest gen sunt in general scumpe, au o logica mascata de detectarea a posibilelor probleme de securitate si sunt greu de inteles si de configurat de catre utilizator dupa propiile nevoi.

Prin oferirea utilizatorilor posibilitatea de a intelege si modifica modul de functionare a unui astfel de sistem poate rezulta in sisteme mult mai eficiente si rapide, dedicate pentru preferintele si nevoile aplicatiei fiacarui utilizator in parte. Spre exemplu, un utilizator poate sa decida ca nu are nevoie de funcionalitaile de detectie impotriva atacurilor de tip SQL injection pentru o anumita aplicatie, intrucat aceasta nu prezinta vulnerabilitati din acest punct de vedere, nefolosind o arhitectura bazata pe baze de date. In cazul acesta prin eliminarea unui astfel de modul, se elimina si verificarile aferente asupra reques-urilor clientilor, imbutatatind astfel performantele sistemului.

\textit{\thesistitle} ofera un sistem configurabil dupa preferintele utilizatorilor. Utilizatorul poate configura detectia bazata pe analiza request-ului primit de la client, acesta poate sa aleaga  care module sa fie folosite pentru detectie, permitand si eventuala adaugare de noi module(cat timp acestea respecta anumite conditii de structura), meniurile din interfata utilizator fiind generate in mod dinamic in functie de modelele prezente. Utilizator poate, de asemenea sa configureze si lista ip-urilior blocate, permitandu-i-se sastearga din cele existente, respectiva sa aduge unele noi, dupa bunul plac.



%\section{Report's Structure}
 \section{Structura lucrării}
In aceasta sectiune se prezinta structura lucrarii pe capitole si o scurta descriere a continutului acestora:

Primul capitol ~\ref{cap:Introducere}, prezinta o scurta introducere despre proiect, contextul acestuia si motivatia pentru implementarea sistemului propus.

Capitolul~\ref{cap:obiective-specificatii} prezinta obiectivele lucrarii, specificatiile sitemului, motivand deciziile luate in implementarea sistemului, cerintele functionale si nonfunctionale necesare implementarii sistemului.

 Capitolul~\ref{cap:studiu-bibliografic} descrie alte abordari similare ale problemelor tratate de proiectul propus, prin evidentierea asemanarilor si diferentelor dintre acestea si se explica tehnologiile si metodele folosite de proiect.
 
In capitolul~\ref{cap:fund-teoretice} sunt evidentiate si explicate pe scurt aspectele teoretice pe care se bazeaza proiectul.

Capitolul~\ref{cap:analiza-si-proiectare} descrie design-ul proiectului si cuprinde: cerintele sistemului, specificatiile cazurilor de utilizare, arhitectura sistemului, comportamentul sistemului, datele utilizate de sistem, dependintele sistemului si algoritmi esentiali si metodele folosite. Descrierea acestora se realizeaza prin asocierea cu diagramelor aferente.


%\chapter{Project's Objectives and Specification}
 \chapter{Obiective și specificații}
\label{cap:obiective-specificatii}

%Acest capitol conține descrierea detaliată a temei de cercetare propriu-zise, formulată exact, cu obiective clare și specificații, pe 2-3 pagini și eventuale figuri explicative. Titlul nu e neapărat impus și, de asemenea, capitolul poate fi inclus ca subcapitol în Capitolul~\ref{cap:Introducere}, dacă se potrivește.
%
%Reprezintă cca. 5--10\% din lucrare.

Acest capitol prezinta obiectivele lucrarii, motivand deciziile luate in implementarea sistemului, specificatiile sitemului, cerintele functionale si nonfunctionale necesare implementarii sistemului.

%\section{Objectives}
 \section{Obiective}
%
%Obiectivele proiectului sunt lucrurile care se dorește a fi realizate, ca urmare a abordării temei lucrării de licență. În general numărul de obiective este proporțional cu timpul de care dispunem. Exemple generice:
%\begin{enumerate}
%  \item Analiza critică a soluțiilor existente pentru problema abordată și identificarea posibile limitări ale acestora.
%  \item Propunerea unor soluții la (o parte) din problemele identificate. 
%  \item Implementarea soluției și validarea ei
%  \item Identificarea unor teme de dezvoltare și cercetare ulterioare
%  \item \dots
%\end{enumerate}
Elaborarea unor masuri de securitate impotriva anumitor tipuri de atacuri sau dezvoltarea unei logici de filtrare a clientilor serviti de catre aplicatie este posibila si in partea de implementare a serverului, cea ce ar putea oferi si o eventuala crestere in performantele aplicatiei, eliminand astfel posibile interceptari suplimentare si validari ale traficului. Insa o astfel de implementare presupune o arhitectura mult mai complexa pentru partea de server si individual cunostinte avansate despre securitate, posibilele vulnerabilitati la care acesta poate sa fie predispus, precum si modalitati de combatere ale acestora. 

O solutie mult mai simpla la aceasta problema este folosirea unui modul care sa implementeze aceste functionalitati separat. Mare parte din produsele de pe piata, ce indeplinesc aceste caracteristici sunt destul de scumpe si nu ofera foarte multa libertate din punctul de vedera al configurari securitatii dorite de catre utilizator asupra produsului sau. In cazul ip-urilor blocate, multe aplicatii nu ofera liberatatea utilizatorilor de a edita listele de referinta ale detectiilor, acetea fiind actualizate automat conform unor date interne, iar eventualele abateri de la aceste reguli se realizeaza prin adaugarea de exceptii. In cea ce priveste validarea request-urilor primite de la clineti, mare parte din aceste sisteme nu ofera o protectie configurabila. Cum sa precizat mai sus, acest tip de sisteme pot sa introduca mici intarzieri datoreate validarilor suplimentare supra request-urilor primite de la clientii produsului, insa in unele cazuri anumite validari sunt facute inutil intrucat produsele respective nuputand sa prezinte vulnerabilitati de acel fel.

 \textit{\thesistitle} are obiectivul de a oferi o componenta gratuita cat mai usor de integrat si de configurat de catre utilizator dupa preferintele produsului sau, care sa facilizeze o protectie cat mai eficienta cu suport pentru imbunatatiri. Sistemtul trebuie sa fie usor de instalat si de configurat, oferindu-i utilizatorului o interfata clara, sugestiva si usor de folosit prin care sa interactioneze cu acesta. Detectiile sistemului trebuie sa fie activabile, utilizator putand alege in momentul  pornirii sistemului ce vulnerabilitati sa fie tratate de acesta. Lista ip-urilor blocate trebuie sa fie usor de vizualizat si editabila, permitand utilizatorului sa isi impuna cu usurinta propiile reguli asupra modului de functionare a sistemului. Pentru realizarea detectiei impotriva atacurilor de tipul SQL injection se impune prelucrarea unor date reale pentru antrenarea modelului de machine learning. Prin folosirea unor date provenite din atacuri reusite sau tentative de atacuri reale, se poate crea o precizie mult mai buna pentru o clasificarea cat mai precisa a posibilelor atacuri. Sistemul trebuie de asemenea sa fie construit modular pentru a permite realizarea de modificari cu usurinta, iar incarcarea detectiilor realizata dinamic, permitand astfel ca adaugarea de noi detectii sa fie cat mai simpla.




%\section{Project Specification}
 \section{Specificații}
\textit{\thesistitle} trebuie sa fie capabil sa serveasca ca si un reverse proxy pentru un server, sa blocheze atacurile de tip SQL injection asupra lui si sa nu permita conectarea clientilor cu ip-uri utilizate de reteaua tor la acesta.

\textit{\thesistitle} va procura utilizatorului o interfata grafica prietenoasa, usor de folosit, prin intermediul careia, acesta va putea sa seteze mediul de rulare al sistemului. Interfata va permite setarea specificatiilor server-ului, adresa si portul pe care acesta accepta conexiuni, dar si a interfetelor prin intermediul carora se pot realiza conexiuni la server. 

In interfata grafica se vor afisa si eventualele detectii realizate de produs. Intr-o fereastra separata utilizatorul trebuie sa aiba posibilitatea sa vizualizeze toate deciziile sistemului si motivele din spatele deciziilor, permitand astfel acestuia sa inteleaga modul de functionare, respectiv sa raporteze sau sa modifice sistemul(in cazurile in care i se ofera aceasta posibilitate) cand comportamentul acestuia nu se afla in conformitate cu nevoile sau cerintele sale.

 In momentul configurarii modului de rulare al sistemului, utilizatorul trebui sa aiba si posibilitatea de a impune ce module de securitate sa fie folosite de acesta in timpul rularii. Pentru a eficientiza cat mai mult sitemul, utilizatorul poate sa aleaga care sunt modulele de interes pentru propria aplicate, evitand astfel validarea unor evenimente ce nu prezinta interes pentru acesta.
 
 
In timpul rularii sistemul va asculta pe interfetele setate de catre utilizator pentu posibile cereri de conexiuni la server-ul setat. In functie de modulele alese in momentul porniri, acesta va verifica sau nu adresa clientului validand astfel conexiunea. In cazul in care adresa clientului se afla pe lista neagra de adrese, conexiunea acestuia este refuzata, iar aplicatia inregistreaza aceasta decizie in fereastra de evenimente vizibila utilizatorului. Dupa ralizarea conexiunii la server, fiecare request trimis de clienti catre acesta va fi evaluat conform modulelor configurate. Daca reqesturile sunt considerate ca fiind "curate" acestea sunt trimise mai departe la server. In caz contrar, clientului i se intoarce un cod de eroare, iar reqestul nu va mai fi trimis mai departe catre server, de asemenea inregistrand evenimentul in fereastra de evenimente vizibila utilizatorului.


\begin{figure}[h]
	\centering
	\includegraphics[width=0.6\textwidth]{fff.png}
	\caption{Cutia neagra a sistemului}
	\label{fig:black-box}
\end{figure}

Figura ~\ref{fig:black-box} prezinta cutia neagra a sistemului propus. \\

%\subsection{Functional Specification}
 \subsection{Specificații funcționale}

Sistemul trebui sa prezinte o interfata grafica usor de folosit de catre utilizator si sa fie capabil sa redirectioneze traficul interceptat catre un anumit server, clasificand si filtrand traficul malitios. Pentru a atinge obiectivele proiectului, urmatoarele cerinte functionale trebuie indeplinite:
\begin{itemize}
  \item Sa realizere conexiunea la un server HTTP/HTTPS si sa redirectioneze traficul primit catre acesta.
  \item Sa intercepteze traficul venit pe o anumita interfata si port prestabilit.
  \item Sa prelucreze request-urile primite de la clineti intr-un format specific clasificatorului de SQL injection.
  \item Sa nu redirectioneze reqesturile clasificate ca si SQL injection.
  \item Sa blocheze conectarea clientilor ce folosesc ip-uri clasificate ca ip-uri de Tor.
  \item Sa permita utilizatorului sa editez si sa vizualizeze lista ip-urilor de Tor.
  \item Sa prezinte in interfata grafica toate interventiile rezlizate asupra traficului(blocari de conexiuni sau de request-uri).
  \item Sa permita utilizatorului sa configureze modul de operare al sistemului.
\end{itemize}


%\subsection{Non-Functional Specification}
 \subsection{Specificații non-funcționale}

Sistemul trebuie, de asemenea, să aibă următoarele caracteristici non-funcționale pentru a realiza obiectivele specificate:
\begin{itemize}
  \item Sa fie usor de instalat si de folosit pentru orice utilizator, oricat de neexperimentat.
  \item Sa poata intercepta traficul de pe orice/oricate interfete disponibile.
  \item Sa poata rula pe orice sistem de operare Windows cu Python2 instalat.
  \item Sa aiba o rata de blocare de 100\% a ip-urilor de pe lista neagra, iar
  in cazul detectiei de SQL injection sa nu aiba detectii false pozitive mai mari 2-3\%
  si o acuratete generala de peste 90\%
\end{itemize}





%\chapter{Bibliographic Survey}
 \chapter{Studiu bibliografic}
\label{cap:studiu-bibliografic}
%
%Documentarea bibliografică are ca obiectiv fixarea referențialului în care se situează tema, prezentarea susrselor bibliografice utilizate și a cercetărilor similare și raportarea abordării din lucrare la acestea.
%
%Referințele bibliografice se vor face pentru fiecare carte, articol sau material folosit pentru elaborarea lucrării de licență. 
%
%Reprezintă cca. 10--15\% din lucrare.
In acest capitol sunt prezentate alte abordari similare ale problemelor tratate de proiectul propus, prin evidentierea asemanarilor si diferentelor dintre acestea si se explica tehnologiile si metodele folosite de proiect.

%\section{Related Work}
 \section{Abordări similare}

%Comparați abordarea  motivand deciziile luate in implementarea sistemului, cu cele ale altor soluții: ce e asemănător, ce e diferit (și, de preferat, mai bun). 
%
%Citarea referințelor se face ca în exemplele \ref{subsec:s10} din Bibliografie. 
%Vezi citările următoare.
%
%În articolul [] autorul descrie configurația tehnică a unei "honeynet" și prezintă câteva atacuri de actualitate asupra honeynet, precum și o serie de recomandări pentru securizarea sistemelor conectate la rețele de calculatoare.

% În capitolul 4 al [], referitor la valoare honeypots, Spitzner prezintă avantajele și dezavantajele acestora.

%În articolul on-line [] găsim detalii interesante despre \dots.


%\section{Technologies and Methods}

Precum Richard Bassett, Cesar Urrutia si 	 Nick Ierace sustin in articolul \textbf{Intrusion prevention systems} \cite{ips} "sistemele de prevenire a intruziunilor sunt o componenta importanata a sistemelor de protectie IT, iar fara aceasta tehnologie, datele noastre si retelele ar fi mult mai predispuse activitatilor malitioase".

In general tentativele de expluatare a vulnerabilitatilor unei aplicatii vin sub forma de input catre o aplicatia tinta. Acest input fiind generat de catre un atacator ce intentioneaza o controleze sau sa ii intrerupa activitatea. In cazul unui atac reusit, un astfel de atacator poate sa dezactiveze temporar aplicatia (atacuri de tipul denial of service) sau poate accesa, altera sau executa date/cod in interiorul aplicatiei. Un sistem de prevenire a intruziunilor are rolul de a examina traficul destinat unei aplicatii si de intercepta si bloca astfel de tentative \cite{what_is_ips}.

Un sistem de prevenire a intruziunilor este, de regula folosit alaturi de un sistem firewall respectiv alaturi de un sistem de detectare a intruziunilor. Desi au scopuri asemanatoarea, aceste sisteme au functionalitati diferite si rezolva diferite probleme de securitate. Un sistem de preventie a instructiunilor este cel mai bine comaprat cu sistemele de tip firewall. Un sistem firewall tipic este constituit dintr-o serie de reguli ce permit traficului sa treaca. Aceste regului sunt sub forma "daca traficul indeplineste anumite conditii poate trece", insa daca nu exista nici o regula care sa potriveasca anumite pachete, acestea sunt blocate. Asemeni sistemelor firewall, sistemele de prevenire a intruziunilor prezinta un set de regului de filtrare a pachetelor, reqest-urilor  sau a clientilor, insa aceste regului sunt de regula reguli de blocare. Astfel, daca un anumit pachet nu potriveste nici o regula sistemul de prevenire a intruziunilor il lasa sa treaca \cite{ips_ids}.

Spre deosebire de sistemele de tip firewall sau cele de prevenire a intruziunilor, care ofera control utilizatorului asupra traficului ce trece prin retea, sistemele de detectie a intruziunilor permite acestuia sa vizualizeze evenimentele din retea. Precum si Joel Snyder sustine in articolul \textbf{Do you need an IDS or IPS, or both?} \cite{ips_ids}  sistemele de detectie a intruziunilor ofera unui utilizator facilitati asemanatoare unui analizator de pachete \cite{net_an}, insa din perspectiva de securitate a retelei. Aceste informatii furnizate de catre sistem ii permit utilizatorului sa decopere: 
 
\begin{itemize}
	\item Incalcari ale politicilor de securitate, precum utilizatori sau siteme ce desfasoara activitati ce incalca politicile prestabilit.
	\item Posibile sisteme infectate ce folosesc reteaua pentru a se raspandi sau sa atace alte sisteme.
	\item Scurgeri de informatie cauzate de infectarea unor sisteme cu malwarei sau de utilizatori rau intentionati.
	\item Erori de configurare in sisteme sau aplicatii cu setari de securitate incorecte sau configurari proaste ce consuma prea multa banda de retea.
	\item Detectarea unor clienti sau servere ce acceseaza sau sunt accesate in mod neautorizat, respectiv aplicatii malitioase ce fac asta.
\end{itemize}
\begin{figure}[h]
	\centering
	\includegraphics[width=0.6\textwidth]{ips.png}
	\caption{Administrarea securitatii unei aplicatii}
	\label{fig:ips-example}
\end{figure}

Figura ~\ref{fig:ips-example} prezinta administrarea securitatii unei aplicatii folosind combinatia dintre cele trei sisteme. \\

In comparatie cu sistemele de detectie a intruziunilor care sunt sisteme pasive si scaneaza reteaua fara sa interferezu cu traficul, sistemele de prevenire a intruziunilor sunt plasate intre server si cilenti, alterand in mod automat traficul in cazul in care acesta declanseaza una din regulile prezente in sistem. Precum sunt prezentate si in articolul \textbf{What is an intrusion prevention system?} \cite{what_is_ips}, printre functionalitatile unui sistem de prevenire a intruziunilor se numara:
\begin{itemize}
	\item Notificarea unui administrator de retea in cazul in care un sau mai multe reguli sunt declansate.
	\item Oprirea pachetelor considerate malitioase pentru retea.
	\item Blocarea unor utilizatori prin excluderea adreselor ip ale acestora.
	\item Resetarea unor conexiuni.
\end{itemize}

In cea ce priveste functionalitatile oferite de un sistem de prevnire a intruziunilor, acestea sunt specifice tipului sistemului. Conform autorului articoluiui \textbf{Intrusion Prevention System (IPS): Definition \& Types} \cite{ips_types}, Beth Hendricks, exista patru tipuri primare de astfel de sisteme:
\begin{itemize}
	\item Network-based: Protejeaza intraga retea.
	\item Wireless: Protejeaza doar reteaua wireless.
	\item Network behavior: Examineaza traficul din retea.
	\item Host-based: Software cu scopul de a proteja un singur calculator.
\end{itemize}

Sistemele de tipul Network-based(reprezentand si categoria in care se incadreaza  \textit{\thesistitle}) presupun implementarea unor senzori in retea carea captureaza si analizeaza traficul ce trece prin acesta. Acesti senzori sunt plasati in puncte cheie a retelei pentru a putea captura in timp reala traficul, iar in cazul interceptarii unor activitati malitioase sa poata interveni imediat, fara sa scada performanta retelei. Aceste siteme ofera protectie retelei indiferent de dimensiunilie sau cresterea acesteia, adaugarea de noi device-uri fiind posibila fara sa necesite adaugarea de noi senziori. Adaugarea de noi senzori fiind nevoita doar in cazul in care traficul retelei depaseste capacitatea de procesarea a senzorilor curenti, infulentand astfel performantele retelei \cite{impl}.

In functie de nevoi, un sistem de prevenire a intruziunilor poate sa ofere diferite optiuni de protectie pentru diferite parti ale retelei. Unele sunt capabile sa opreasca traficul malitios sau sa limiteze latimea de banda catre anumite parti ale retelei. Conform \cite{ips_sec_types} aceste siteme pot oferi protecti impotriva urmatoarelor tipuri de atacuri:
\begin{itemize}
	\item \textbf{ICMP Storms:} un volum mare de ecouri ICMP pot sa indice activitati malitioase precum cineva ce scaneaza reteaua.
	\item \textbf{Ping to Death:} un utilizator poate sa modifice comanda de ping, astfel incat sa trimita un numar mare de pachete de dimensiune mare catre o destinatie tinta pentru a o scoate din uz.
	\item \textbf{SSL Evasion:} unele atacuri se pot folosi de criptarea SSL pentru a evita dispozitivele de securitate, intrucat in general acestea nu sunt decriptate.
	\item \textbf{IP Fragmentation:} consta in expluatarea faptului ca pachetele sunt descompuse in fragmente pentru a staisface cerintele de dimensiune a retelelor traversate, inundand o destinatie tinta cu fragmente false pentru a ii consuma resursele. 
	\item \textbf{SMTP mass mailing attacks:} un sistem infectat poate sa se foloseasca de de email-ul utilizatorului pentru a se raspandi, rezultand intr-un trafic mare destinat serverului de mail.
	\item \textbf{DoS/DDoS attacks:} cu scopul de a face o resursa indisponibila utilizatorilor, este realizata prin inundarea sistemului tinta cu un numar mare request-uri de la unul sau mai multe(in cazul DoS distribuit - DDoS) siteme malitioase.
	\item \textbf{SYN Flood attacks:} atacatorul trimite un numar mare de pachete de initiare a unei conexiuni fara sa mai raspunda ulterior, epuizand astfel resursele de memorie.
	\item \textbf{Http obfuscation:} pentru a evita sa fie detectate de anumite siteme de protectie, unele atacuri folosesc tehnici de ofuscare a request-urilor HTTP.
	\item \textbf{Port Scanning:} este folosit pentru descoperi ce porturi sunt deschise pe un sistem, ulterior permitandu-i atacatorului sa stie ce vulnerabilitati ar putea prezneta sistemul.
	\item \textbf{ARP Spoofing:} un atacator trimite in retea pachete false de ARP legandu-si propria adresa MAC de adresa IP a unui alt sistem. Ca urmare, atacatorul va primi pachete destinate sistemului cu adresa IP folosita in pachetul de ARP.
	\item \textbf{CGI Attacks:} un atacator poate sa trimita request-uri malitioase, determinand destinatia sa trateze request-ul primit ca si cod executabil, oferindu-i atacatorului acces pe sistem.
	\item \textbf{Buffer Overflow attacks:} presupune ca atacatorul sa depaseasca limitele unui buffer de lungime fixa, excesul de date ajungand sa suprascrie zone adiacente de memorie rezultand in caderea sistemului sau dandu-i atacatorului oportunitatea sa ruleze cod propriu.
	\item \textbf{OS Fingerprinting attacks:} presupune ca atacatorul sa descopere ce sistem de operare ruleaza pe un sistem si folosindu-se de aceasta informatie sa expluateze vulnerabilitati specifice acelui sistem de operare.
\end{itemize}

Sistemul propus, \textit{\thesistitle} implementeaza un sistem de prevenire a intruziunilor folosindu-se de un reverse proxy pentru a intercepta tot traficul care intra si iese din retea( reprezentand senzorii ce au rol de a captura si analiza traficul) si oferind protectie impotriva a doua categorii de atacuri: SQL injection si blocarea traficului venit de la ip-uri ce utilizeara frecvent Tor.

Pentru prevenirea atacurilor de SQL injection, se foloseste o metoda asemanatoare celei propuse de Eun Hong Cheon, Zhongyue Huang si Yon Sik Lee in lucrarea \textbf{Preventing SQL Injection Attack Based on Machine Learning} \cite{sqli_how}. Pentru clasifiacrea request-urilor HTTP in SQL injection sau curate, se foloseste un sistem bazat pe machine learning. Acest sistem este antrnat anterior cu date reale, ca si trasaturi fundamentale in clasificare, folosindu-se cuvintele cheie si simbolurile specifice limbajului SQL(spre exemplu: SELECT, ADD, DELETE, ", ' etc.).

\begin{figure}[h]
\centering
\includegraphics[width=0.6\textwidth]{sqli.png}
\caption{Tipuri de trasaturi ale limbajului SQL}
\label{fig:sql-features}
\end{figure}

Figura ~\ref{fig:sql-features} prezinta tipurile de trasaturi luate in considerare in lucrarea \textbf{Preventing SQL Injection Attack Based on Machine Learning} in raport cu simbolurile sau cuvintele cheie folosite. 


O alta abordare pentru prevenirea atacurilor SQLI este propusa de Fredrik Valeur, Darren Mutz si Giovanni Vigna in lucrarea \textbf{A Learning-Based Approach to the Detection of SQL Attacks} \cite{sqli_how2}. In aceasta lucrarea se prezinta folosirea unui sistem bazat pe detectia de anomalii pentru detectarea atacurilor ce expluateaza o aplicatie pentru a ii compromite baza de date. Asemeni abordarii bazate pe machine learning, acest sistem presupune o faza anterioara de antrenare in care se invata comportamentul normal al utilizatorilor, alcatuind astfel niste profile specifice. Astfel in faza de detectie, comportamentul ce nu coincide cu profilele alcatuite in faza de antrenate, este considerat malitios.

Pentru prevenirea utilizatorilor de Tor, in general lista de ip-uri este alcatuita din toate ip-urile care au utilizat reteaua intr-un anumit interval de timp, aceasta fiind actualizata periodic. O astfel de abordare este folosita si in cazul marelui firewall al Chinei \cite{china_tor} care indentifica nodurile la prima accesare a retelei de Tor. Insa precum precum se evidentiaza si in articolul \textbf{Characterizing the Nature and Dynamics of Tor Exit Blocking} \cite{tor_1} o astfel de abordare nu este cinstita fata de unii utilizatori de Tor, intrucat reputatia acestora este impartita intre toti utilizatorii. Astfel un nod care este utilizat doar pentru cateva minute sau ore(probabil din motive de curiozitate) poate sa ajunga sa fie blocat, fiind tratat asemeni cu un nod ce functionaza de cateva zile. O astfel de discriminare a incercat sa fie evitata prin implementarea aleasa a sistemului propus. Pentru realizarea listei de ip-uri blocate se foloseste un algoritm ce stabileste o limita de timp minima de functionare pentru un anumit nod in intervalul a 30 de zile.


 \section{Tehnici/Tehnologii/Surse folosite}

Pentru realizarea sistemului propus s-au folosit doua limbaje de programare: Pyhton2/3(pentru partea de back end) si C\#(pentru partea de front end). In partea de back end a proiectului se realizeaza implementarea unui reverse proxy pentru a intercepta traficul uneia sau mai multor interfete, un modul pentru clasificarea request-urilor impotriva atacurilor SQL injection si un modul pentru generarea listei negre de ip-uri ce utilizeaza frecvent reteaua Tor. Toate aceste componente sunt realizate prin utilizarea de librarii open-source pentru a usura si eficientiza munca precum: twisted \cite{twisted}, beautiful soup \cite{btf_soup}, libsvm \cite{libsvm}.

Motivul utilizarii atat limbajului Python3 cat si Python2 este datorat diferenteleor de module si librarii open-source disponibile pentru cele doua limbaje dar si a fatului ca suportul pentru Python2 se inchei in anul 2020. Conform documentatiilor oficiale \cite{python3_doc} si \cite{python2_doc}, dar si articolului \textit{Python 2 to python 3: why, and how hard can it be?} de Tim Grey \cite{why_python3}, intre cele doua versiuni nu sunt modificari majore, insa in anumite cazuri pot exista librarii care sa ofere doar suport pentru una dintre aceasta.

In realizarea modulului pentru clasificarea request-urilor impotriva atacurilor SQL injection sa folosit o colectie de date reale atat de atacuri cat si de trafic curat. Pentru uniformizarea acestor date si pentru a trata tentativele de pacalire a clafisicatorului prin codarea unor caractere in valoarea lor in cod hexadecimal (exemplu 'https://www.google.ro /search?q=a' echivalent cu 'https://www.google.ro/search?q=\%61') datele au fost preprocesate si transformate in intregime in coduri hexadecimale \cite{ascii}. In procesarea datelor, petru indetificarea trasaturilor relevant, s-au indentificat caracterele specifice limbajului \cite{char_sql} si cuvintele cheie rezervare \cite{key_sql}. Ulterior, pentru antrenarea modelului de support vector machine si pentru clasificarea noilor request-uri s-a folosit software-ul open-source libsvm \cite{libsvm_class}. 

\begin{figure}[h]
	\centering
	\includegraphics[width=0.6\textwidth]{sql_arh.png}
	\caption{Arhitectura unui sistem de clasificare a request-urilor HTTP}
	\label{fig:sql-arh}
\end{figure}

Figura ~\ref{fig:sql-arh} arhitectura unui sistem de clasificare a request-urilor HTTP de un sistem bazat pe machine learning. Structura este prezentata in lucrarea prezentata si anterior \textbf{Preventing SQL Injection Attack Based on Machine Learning} \cite{sqli_how}. Acesta structura a reprezentat un model de pornire in realizarea modulului de prevenire a atacurilor SQL injection, implementarea modulului incercand sa aduca imbunatatiri de performanta prin modificarea algoritmului folosit pentru antrenarea modelului de support vector machine dar si prin filtrarea trasaturilor propus in lucrare in conformitate cu raportul dintre obijnuinta de aparitie a acestora atat in request-urile ce intentioneaza sa execute un atac cat si in cele curate.

Pentru blocarea ip-urilor utilizate de reteua Tor s-a folosit un script scris in Python3. Programul interogheaza periodic(din 6 in 6 ore) informatiile oferite de \textit{Tor Network Status} \cite{tot_status} indentificand astfel nodurile cu un "Uptime" mai mare de 7 zile in parcursul unei luni. Blocarea ip-urilor se realizeaza prin compoararea cu o astfel de lista generata lunar.

Componenta ce incorporeaza toate modulele de protectie, este cea de reverse proxy. Aici este monitorizat tot traficul ce vine de pe o anumita interfata(una sau mai multe, in functie de configuratia utilizatorului) si este trecut prin toate modulele disponibile pentru a verifica conditiile de securitate. Pentru testarea daca o adresa ip este utilizata frecvent de reteaua Tor, in momentul in care un client doreste sa realizaze o conexiune la server-ul protejat de sistem, adresa ip a acestuia este verificata sa nu se afle pe lista ip-urilor blocate. Pentru actualitate, lista adreselor ip blocate este actualizata periodic cu adresele ip utilizate frecvent de reteaua Tor in ultima luna. Modulul de prevenire a atacurilor SQL injection este integrat tot in componenta de reverse proxy, insa evaluarea request-urilor este facuta dupa realizarea conexiunii intre client si server. Request-urile primite de catre server sunt tratate asemanator celor folosite pentru antrenarea modulului de support vector machine, insa pentru clasificarea acestora este folosit modulul antrenat in faza initiala si software-ul de prezicere oferit tot de libsvm \cite{libsvm}.
	
	

	






% \chapter{Theoretical Backgound}
\chapter{Fundamente teoretice}
\label{cap:fund-teoretice}


In acest capitol sunt evidentiate si explicate pe scurt aspectele teoretice pe care se bazeaza proiectul.

%
%Aici se descriu pe scurt aspecte teoretice pe care se bazează lucrarea. Conținutul acestui capitol trebuie gândit pentru un citor care nu e specializat pe domeniul temei și nu cunoaște chestiunile de bază despre subiect. Pentru un cititor specializat, capitolul poate să stabilească un limbaj comun, relativ la termenii care pot fi interpretați diferit. 
%
%Acest capitol nu trebuie gândit și scris nici ca un copy-paste din alte surse, nici ca zona de reglaj a numărului de pagini ale lucrării. Deși va conține chestiuni pe care le-ați studiat și voi și pe care v-ați bazat, el trebuie să fie o compilare a surselor folosite, care să aibă sens și relevanță pentru lucrarea voastră. Trebuie să fie o descriere coerentă și logică a unor aspecte care ușurează sau fac posibilă înțelegerea părților următoare ale lucrării. Nu trebuie intrat insă prea mult în detalii, ci spuse doar chestiunile esențiale. 
%
%Dacă preluați text, figuri, tabela etc. din sursele de documentare, acestea din urmă trebuie indicate explicit. 
%
%Reprezintă cca. 10--15\% din lucrare.
\section{Reverse proxy}

Un reverse proxy este un server intermediar care trimite mai departe request-urile pentru continut, de la mai multi clienti nedefiniti, catre unul sau mai multe servere. Un reverse proxy este un tip de proxy care in mod normal este situat in spatele unui firewall intr-o retea interna si redirectioneaza traficul clientilor catre serverele asociate. Acesta introduce un nivel in plus de abstractizare si control, asigurand controlul fluxului de trafic \cite{rev_proxy_server}.

\begin{figure}[h]
	\centering
	\includegraphics[width=0.6\textwidth]{rev-proxy.png}
	\caption{Folosirea unui reverse proxy in arhitectura unei aplicatii.}
	\label{fig:rev-proxy}
\end{figure}

Figura ~\ref{fig:rev-proxy} prezinta modalitatea de integrare a unui reverse proxy in implementarea arhitecturii de back end a unei aplicatii. \\

Cele mai obisnuite caracteristici ce pot fi oferite de utilizarea unui reverse proxy sunt:
\begin{itemize}
	\item Load balancing - un reverse proxy poate sa distribuie request-urile primite de la clienti, astfel incat nici un server sa nu fie coplesit ce reqesturi. In cazul in care un server este supraincarcat cu reqest-uri sau este cazut, acesta poate sa redirectioneze traficul carea alte servere functionale.
	\item Web acceleration - un reverse proxy poate sa realizeze compresia datelor sau sa memoreze in memoria cache continut ce este frecvent accesat sau poate sa realizeze operatiile de criptare SSL executate in mod normal de server, imbunatatind astfel in mod considerabil viteza de comunicare dintra client si serverul destinatie.
	\item Securitate si anonimitate - prin interceptarea request-urilor primite de catre server, acesta asigura anonimitatea serverului actionand ca un nivel extra de securitate. De asemenea se asigura ca mai multe servere pot fi accesate prin intermediul unui punct comun, indiferent de structura retelei interne.
\end{itemize}



\section{Support vector machine}

Algoritmul de machine learning, support vector machine reprezinta un model obtinut prin folosirea de diversi algoritmi pentru antrenarea acestuia, folosit pentru a clasifica date. Acest model intra in categoria de invatare supravegheata('supervised learning'), intrucat pentru obtinerea lui se foloseste un set de date ca si exemplu, date pe care modelul le va folosi ca referinta pentru clasificarea noilor evenimente.

Realizarea unui astfel de model se obtine in urma executarii unui proces elaborat ce implica mai multi pasi:
\begin{itemize}
	\item  Primul pas reprezinta indentificarea datelor relevante in cea ce priveste problema tratata(setul de antrenare). In conformitate cu scopul clasificarii unor evenimente/ date, in doua(sau mai multe) categorii, initial trebuie indentificate o serie de astfel de evenimente si categorizate de catre utilizator in evenimente ce sigur apartin fiecarei dintre categoriile tinta.
	\item Dupa  obtinerea datelor de antrenare, trebuie indentificate toate trasaturile relevante din aceste date, trasaturi care sa fie cat se poate de specifice fiecarei categorii in parte. Fiind recomandata evitarea trasaturilor ce sunt prezente in mare parte din date sub aceasi forma (ex:caracterul '=' sau'?' intr-un URI folosit pentru clasificarea atacurilor SQL injection), indiferent de categoria din care acestea fac parte.
	\item Dupa obtinerea trasaturilor specifice datelor de antrenare, se realizeaza antrenarea modelului folosind un algoritm specific. In cazul proiectului propus s-a folsoit algoritmul gata implementat, furnizat de biblioteca open source LIBSVM \cite{libsvm}. Pentru obtinerea modelului, datele de antrenare au fost procesate folosind un kernel gausian. Un kernel gausian reprezinta modul in care modelul proceseaza datele de antrenare astfel incat clasificarea noilor date sa fie realizata prin calcularea similaritatilor dintre acestea si cele de antrenare. In calcularea similaritatii dintre aceste doua tipuri de date, un parametru foare important este sigma. Acest parametru este ales pentru intrg setul de date, iar valoarea lui este diret proportionala cu gradul de similaritate pe care algoritmul il va asocia la doua evenimente/date diferite.
\end{itemize}



\begin{figure}[h]
	\centering
	\includegraphics[width=0.8\textwidth]{svm-andrew.png}
	\caption{Influentele aduse algoritmului de modifiacrea parametrului sigma in algoritmul de antrenare.}
	\label{fig:rev-proxy}
\end{figure}


Figura ~\ref{fig:rev-proxy} prezinta cum inflenteaza clasificarea unui nou eveniment valoarea parametrului sigma din formula algoritmului de support vector machine. Figura ~\ref{fig:rev-proxy} a fost preluata din slide-urile cursului de machine leraning sustinut de Andrew Ng \cite{andrew_ng} \\


\section{SQL injection}

Atacurile de tipul SQL injection sunt realizate prin injectarea de cod executabil intr-o baza de date.

Procesul de interactionare cu o baza de date presupene realizarea de interogari asupra acesteia. In formularea acestor interogari, utilizatorul trebuie sa prezinte interpretorului, sub forma de siruri de caractere, numele tabelelor interogate sau valorile unor capuri specifice din acestea. Aceste siruri de caractere sunt delimitate folosind caracterul " sau '. Atacurile de tipul SQL injection expluateaza folosirea acestor delimitatori de siruri de caratere, trimitand siruri de caractere eronate intentionat catre baza de date. Un utilizator rau intentionat poate sa furnizeze astfel de siruri de caractere catre o baze de date prin intermediul oricarui procesator de continut disponibil unui client al unei aplicatii ce comunica cu o baza de date. Aceste siruri de caractere delimiteaza prematur valoarea care este folosita in interogare, introducand dupa aceasta o serie de caractere pe care interpretorul le va trata ca si cod executabil, oferindui astfel utilizatorului sa execute operatiuni asupra bazei de date la care nu ar avea accesul in mod normal. Aceste operatiuni pot sa reprezinte alterarea bazei de date sau obtinerea de date confidentiale. \\



\begin{figure}[h]
	\centering
	\includegraphics[width=0.4\textwidth]{259px-Sql_Injection_Login.png}
	\caption{Exemplu de atac realizat prin SQL injection.}
	\label{fig:sqli-example}
\end{figure}

Figura ~\ref{fig:sqli-example} prezinta o tenatativa de atac prin SQL injection in care in campul de validare a email-ului se incearca injectarea de cod ce va fi executat in interogarea de validare a credentialelor. Prin prezenta caracterului ' se escapeaza tot textul urmat dupa acesta ca fiind cod si nu un string ce face parte din campul de email. Operatia logica "OR 1=1" va determina interpretorul sa returneze adevarat(valid) pentru orice adresa de email introdusa introdusa inaintea caractrului '. \\


\section{Adresele IP ale retelei Tor}
Reteaua Tor reprezinta un softrware gratis de anonimizare a traficului pe internet. Numele este de fapt un actonim pentru "The Onion Router"(router-ul ceapa) care sugereaza modul de functionarea al acestuia, ficare nod din retea adugand un strat extra de securitate celor precedente. Modul de functionare al retelei se bazeaza pe rutarea traficului prin cat mai multe noduri pentru a anonimiza si a face cat mai greu de urmarit traficul unei anumite persoane. Aceste noduri prin care traficul este directionat sunt sustinute gratis de catre voluntari/utilizatori de Tor din intreaga lume.

Pentru criptarea traficului reteaua Tor foloseste criptarea la nivelul aplicatiei in cea ce priveste sructura retelelor de calculatoare(modelul OSI sau TCP/IP). Datele trasmise sunt criptate, incluzand destinatia, cu exceptia nodului urmator, astfel creandu-se structura de "ceapa" asupra unui pachet. Selectia nodurilor prin care se face rutarea pachetelor este aleasa random. Fiecare pachet decripteaza un "strat", afand nodul urmator pentru pachetul respectiv, nodul final decriptand datele initiale si realizand trasmisia catre destinatie, fara sa ii comunice sursa pachetului. Intrucat in comunicarea pachetelor, pe parcursul rutelor parcurse, se cunoaste in permanenta doar nodul urmator, acest lucru impiedica monitorizarea traficului intre sursa si destinatie.


Cu toate ca reteua tor ofera anonimitate de partea clientului, acest lucru nu se realizeaza si fata de ea. Reteaua nu se ascunde fata de serviciile acesate prin intermediul ei. Astfel un site anume poate sa detecteze daca un anumit client il acceseaza folosind reteaua Tor sau nu.

Intrucat reteaua Tor nu isi ascunde adresele IP folosite de catre aceasta, indentificarea lor si procurarea de date despe acestea este destul de usoara. In proiectul propus s-a folosit un serviciu care furnizeaza gratui astfel de date \cite{tot_status}(adresa IP, uptime etc.) si s-au folosit algoritmi propri pentru procesarea acestor date, eliminand astfel necorectitudinea dintre utilizatorii retelei.

\begin{figure}[h]
	\centering
	\includegraphics[width=0.6\textwidth]{can-you-hide-on-Tor-Network.png}
	\caption{Exemplu de trafic realizat prin reteaua Tor.}
	\label{fig:tor-example}
\end{figure}

Figura ~\ref{fig:tor-example} prezinta principalele elemente folosite la rutarea treficului de la client la destinatie prin intermediul retelei Tor. \\


\section{Sistem de prevenire a intruziunilor}
Conform scurtei descrieri prezentate in capitolul anterior, un sistem de prevenire a intruziunilor are rolul de a filtra traficul dintre clientii unui server si serverul propiu zis. 

Acest sistem functioneaza liniar, adica este plasat direct intre server si clienti acestuia. In cazul proiectului propus, componenta de baza pentru interceptarea traficului este realizata prin implementarea unui reverse proxy, oferind astfel caracteristica de interceptare si decriptare a traficului, ce permite analiza acestuia, dar si avantajele specifice utilizarii unui reverse proxy.

Pentru filtrarea traficului, un sistem de prevenire a intruziunilor implementeaza anumiti senzori care au rolul sa inspecteze tot traficul ce trece prin sistem, realizand aceasta inspectie in timp real. Datorita acestei verificari, orice pachet considerat malitios este oprit din a ajunge la serverul destinatie. In proiectul propus, implementarea acestor senzori este realizata in doua moduri. In cazul validarii adreselor IP impotriva utilizatorilor de Tor se folosete o lista de IP-uri ce contine adrese frecvent utilizate de reteaua Tor. In interiorul reverse proxy-ului, in momentul crearii unei noi conexiuni, acesta verfica ca adresa IP ce solicita conectarea la server sa nu fie continuta de lista mentionata. In cazul detectiei impotriva atacurilor de SQL injection, senzorul este implementat utilizand un model de support vector machine. In interiorul reverse proxy-ului in momentul interceptarii unui request venit din exterior catre reteaua interna, acesta verifica daca reqestul poate fi clasificat ca si tentativa de atac. In caz afirmativ blocand trecerea acestuia mai departe catre server.
\begin{figure}[h]
	\centering
	\includegraphics[width=0.6\textwidth]{IPS.png}
	\caption{Integrarea unui sistem de prevenire a intruziunilor intr-o retea.}
	\label{fig:ips-2nd-example}
\end{figure}

Figura ~\ref{fig:ips-2nd-example} prezinta arhitectura unei retele interne ce integreaza un sistem de prevenire a intruziunilor pentru protejarea acesteia. \\

Un sistem de prevenire a intruziunilor poate sa efectueze oricare din urmatoarele actiuni in momentul detectarii unui eveniment malitios \cite{ips_fire}:
\begin{itemize}
%	Terminates the TCP session that is being exploited by an outsider for the attack. It blocks the offending user account or source IP address that attempts to access the target host, application, or other resources unethically.
	\item Sa intrerupa sesiunea dintre client si server, in cazul in care clientul desfasoara sau incearca sa desfasoare activitati malitioase in reteaua protejata de sistem. Acest lucru se poate realiza prin blocarea anumitor credentiale asociate cu utilizatorul respectiv sau prin blocarea adresei IP a acestuia.
	\item In conditiile in care un sistem de prevenire a intruziunilor detecteaza/clasifica o activitate ca fiind malitioasa, acesta poate sa ia masuri automat pentru a preveni un astfel de atac pe viitor(ex: in momentul detectarii unei tentative de atac prin SQL injection, sistemul de prevenire a intruziunilor poate sa blocheze in mod automat adresa IP a utilizatorului ce inceraca sa faca atacul, nepermitandu-i acestuia sa se mai conecteze la serverul destinatie pentru un anumit interval de timp sau pana la interventia unui administrator).
	\item O alata abordare posibila in momentul declansarii unui eveniment malitios este alterarea traficului astfel incat sa elimine continutul malitios din acesta. Pentru realizarea acestui lucru, un sistem de prevenire a intruziunilor poate sa stearga atasamente infectate din interiorul unui mail, sa altereze continutul unui pachet sau sa omita trasmiterea mai departe a unor pachete.
	
\end{itemize}


%\chapter{Analysis and Design}
\chapter{Analiză și proiectare}
\label{cap:analiza-si-proiectare}


\section{Cerintele sistemului}




Sistemul trebui sa indeplineasca urmatoarele cerinte \textbf{functionale}:
\begin{enumerate}
	\item Sa realizere conexiunea la un server HTTP/HTTPS si sa redirectioneze traficul primit catre acesta.
	\item Sa intercepteze traficul venit pe o anumita interfata si port prestabilit.
	\item Sa prelucreze request-urile primite de la clineti intr-un format specific clasificatorului de SQL injection.
	\item Sa nu redirectioneze reqesturile clasificate ca si SQL injection.
	\item Sa blocheze conectarea clientilor ce folosesc ip-uri clasificate ca ip-uri de Tor.
	\item Sa permita utilizatorului sa editez si sa vizualizeze lista ip-urilor de Tor.
	\item Sa prezinte in interfata grafica toate interventiile rezlizate asupra traficului(blocari de conexiuni sau de request-uri).
	\item Sa permita utilizatorului sa configureze modul de operare al sistemului.
\end{enumerate}

Sistemul trebuie, de asemenea, să aibă următoarele caracteristici \textbf{non-funcționale}:
\begin{enumerate}
	\item Sa fie usor de instalat si de folosit pentru orice utilizator, oricat de neexperimentat.
	\item Sa poata intercepta traficul de pe orice/oricate interfete disponibile.
	\item Sa poata rula pe orice sistem de operare Windows cu Python2 instalat.
	\item Sa aiba o rata de blocare de 100\% a ip-urilor de pe lista neagra, iar
	in cazul detectiei de SQL injection sa nu aiba detectii false pozitive mai mari 2-3\%
	si o acuratete generala de peste 90\%
\end{enumerate}

%Acest capitol descrie design-ul proiectului și cuprinde, în general: 
%\begin{enumerate}
%  \item ilustrarea arhitecturii generale și detaliate a sistemului implementat, care să evidențieze modulele componente și relațiile dintre acestea
%  \item stările prin care trece sistemul în decursul funcționării sale (diagrame de stare)
%  \item modul de interacțiune dintre module și funcționalitatea acestora ilustrată prin diagrame de secvențe
%  \item descrierea algoritmilor/metodelor pe care se bazează funcționarea sistemului dezvoltat
%  \item descrierea organizării/structurii eventualelor baze de date folosite
%  \item justificarea alegerilor/deciziilor făcute și analiza critică a acestora (avantaje și dezavantaje), prin comparație cu alte alternative posibile
%\end{enumerate}
%
%Ca idee generală, design-ul trebuie să fie prezentat independent de o implementare anume, în general, și de cea a voastră, în particular. De asemenea, descrierea design-ului trebuie să conțină toate elementele și detaliile necesare, astfel încât altcineva decât voi să poate realiza o implementare a lui, fără a fi nevoit să ia decizii arhitecturale sau organizare (adică, de design) și să vă contacteze pentru a-și lămuri anumite aspecte neclare.
%
%Capitolul trebuie organizat pe secțiuni și subsecțiuni astfel descrierea să urmeze un cors logic și ușor de urmărit. 
%
%Ponderea acestui capitol relativ la întreaga lucrare este de 25-35\%.
%
%
%\section{Examples: lists, figures, tables, equations}
%
%Așa arată o listă de elemente nenumerotate:
%\begin{itemize}
%  \item element 1
%  \item element 2
%  \item \dots
%\end{itemize}
%
%
%Așa arată o listă de elemente numerotare:
%\begin{itemize}
%  \item element 1
%  \item element 2
%  \item \dots
%\end{itemize}
%
%
%Așa arată o listă în text: 
%\begin{inparaenum}[(\itshape 1 \upshape)]
%  \item element 1, 
%  \item element 2, 
%  \item \dots
%\end{inparaenum}
%
%\textbf{Atenție}: orice tabel, figura sau ecuație (formulă) trebuie referite \textit{explicit} în text explicit (de genul: în Figura X este ulustrat \dots, în Tabelul Y se poate vedea \dots), pentru că Latex le poate plasa chiar și pe altă pagină decât acolo unde vrem noi să ne referim la ele. Vedeți exemple de mai jos!
%
%Tabelul~\ref{table:example} ilustrează un exemplu de tabel. Un editor on-line de tabele poate fi găsit la \url{http://www.tablesgenerator.com/}. 
%
%\begin{table}[t]
%\centering                          % tabel centrat 
%\begin{tabular}{|c|c|c|c|}          % 4 coloane centrate 
%\hline\hline                        % linie orizontala dubla
%Case & Method\#1 & Method\#2 & Method\#3 \\ [0.5ex]   % inserare tabel
%%heading
%\hline                              % linie orizontal simpla
%1 & 50 & 837 & 970 \\               % corpul tabelului 
%2 & 47 & 877 & 230 \\
%3 & 31 & 25 & 415 \\[1ex]           % [1ex] adds vertical space
%\hline                              
%\end{tabular}
%\caption{Nonlinear Model Results}   % titlul tabelului
%\label{table:example}                % \label{table:nonlin} introduce eticheta folosita pentru referirea tabelului in text; referirea in text se va face cu \ref{table:nonlin}
%\end{table}
%
%În Figura~\ref{fig:exemplu} 
%
%\begin{figure}
%    \centering
%    \includegraphics[width=0.5\textwidth]{image}
%    \caption{Numele figurii}
%    \label{fig:exemplu}
%\end{figure}
%
%
%Formula~(\ref{eq:example}) arată modul de calcul al lui $\Delta$:
%\begin{equation} \label{eq:example}
%    \Delta =\sum_{i=1}^N w_i (x_i - \bar{x})^2 .
%\end{equation}
%
%
%Algoritmul~\ref{alg:example} este un exemplu de descriere pseudo-cod a unui algoritm, preluat de la \href{http://en.wikibooks.org/wiki/LaTeX/Algorithms#Typesetting_using_the_algorithm2e_package}{http://en.wikibooks.org/wiki/LaTeX}. El utilizează pachetul \textit{algorithm2e}. Alternativ, puteți utiliza pachetele \textit{algorithmic} sau \textit{program}. 
%
%\begin{algorithm}
% \KwData{this text}
% \KwResult{how to write algorithm with \LaTeX2e }
% initialization\;
% \While{not at end of this document}{
%  read current\;
%  \eIf{understand}{
%   go to next section\;
%   current section becomes this one\;
%   }{
%   go back to the beginning of current section\;
%  }
% }
% \caption{How to write algorithms}
% \label{alg:example}
%\end{algorithm}

%\chapter{Implementation Details}
 \chapter{Detalii de implementare}
\label{cap:implementare}


\section{Structura codului sursa}

Codul folosit pentru obtinerea sistemului \textit{\thesistitle} este impartit in 3 proiecte separate:
\begin{itemize}
	\item \textbf{Tor activity monitor} ce constituie logica necesara obtinerii listei de adrese IP blocate de sistem.
	\item \textbf{SQLi SVM} scopul acestuia fiind obtinerea modelului de SVM folosit pentru prevenirea atacurilor de SQL injection.
	\item \textbf{\thesistitle} reprezantand sistemul propus si incorporeaza rezultatele obtinute de celelalte doua proiecte.
\end{itemize}

\begin{figure}[h]
	\centering
	\includegraphics[width=0.8\textwidth]{tor_activity_monitor.png}
	\caption{Modulul pentru monitorizatea activitatii retelei Tor}
	\label{fig:tor_activity_monitor}
\end{figure}

Figura ~\ref{fig:tor_activity_monitor} prezinta care sunt fisierele folosite pentru monitorizarea activitatii retelei Tor si pentru obtinerea listei cu adresele IP ce trebuie blocate de catre sistem si relatiile dintre acestea. \\

\textbf{run\_Get\_new\_Tor\_ips.sh} este un fisier de bash ce are rolul de a rula Get\_new\_Tor\_ips.py. Fisierul este programat sa lanseze in executie Get\_new\_Tor\_ips.py la ore fixe, acesta ruland incontinuu pe un sistem cu acces la internet neintrerupt. Lnasarile in executie au loc o data la 6 ore, respectiv la ora 12 am si pm si 6 am si pm.

\textbf{Get\_new\_Tor\_ips.py} are rolul de a documenta modificarile de uptime din ultimele 6 ore ale nodurilor retelei Tor. Datele sunt furnizate de pe pagina "Tor Network Status" \cite{tot_status} si pentru fiecare adresa IP prezenta in date, se verifica care este uptime ul din ultimele 6 ore, informatiile acestea fiind stocate in IPs\_Tor\_activity.txt.

\textbf{IPs\_Tor\_activity.txt} are rolul de a stoca informatiile despre toate adresele IP utilizate de reteaua tor din ultima luna. Acestea sunt salvate in liste, pentru fiecare adresa IP in parte o lista. 

\textbf{Get\_ips\_blacklist.py} are rolul de genera fisierul Tor\_ips.txt folosit de catre sistemul propus pentru blocarea adreselor IP. Acesta foloseste datele din interiorul fisierului IPs\_Tor\_activity.txt, pentru a genera o lista cu toate adresele IP ce au un uptime total mai mare de 7 zile in ultimele 30 de zile.

\textbf{Tor\_ips.txt} reprezinta rezultatul proiectului. Acesta este alcatuit dintr-o lista formata din toate adresele IP ce vor fi blocate de sistemul propus in timpul rularii.

\newpage

\begin{figure}[h]
	\centering
	\includegraphics[width=0.8\textwidth]{sqli_svm.png}
	\caption{Modulul pentru prevenirea atacurilor SQL injection}
	\label{fig:sqli_svm}
\end{figure}
Figura ~\ref{fig:sqli_svm} prezinta care sunt fisierele utilizate pentru procesarea datelor necesare in procesul de antrenare a modelului de SVM, pana la obtinerea modelului propiu-zis si care sunt realtiile dintre acestea. \\

\textbf{SQLI\_URLs} reprezinta setul initial de date "infected" folosite pentru indentificarea atacurilor SQL injection. Acest set este constituit din URL-uri ce au fost indentificate de un produs autorizat ca fiind tentative de SQL injection.

\textbf{Clean\_URLs} reprezinta setul initial de date "clean" folosite pentru indentificarea atacurilor SQL injection. Acest set este constituit din URL-uri ce au fost indentificate de un produs autorizat ca fiind URL-uri curate.

\textbf{URL\_processor.py} primeste ca input un fisier sau string si are rolul de a procesa URL-uri intr-un format uniform(se elimina encodarile) si specific pentru pasii urmatori.

\textbf{SQLI\_URIs} constituie noua lista rezultata din procesarea fisierului SQLI\_URLs de catre URL\_processor.py.

\textbf{Clean\_URIs} constituie noua lista rezultata din procesarea fisierului Clean\_URLs de catre URL\_processor.py.

\textbf{Get\_secial\_chars\_frequency.py} are rolul de a caldula frecvventa de aparite a unor caractere speciale in cele doua seturi de date si de a decide in functie de frecventa lor de aparite, care din acestea sunt relevante in vederea alegerii trasaturilor de clasificare.

\textbf{key\_features} este o lista alcatuita din toate cuvintele cheie a limbajului SQL, dar si din caracterele speciale utilizate in acesta si considerate ca fiind relevante in urma executiei scriptului Get\_secial\_chars\_frequency.py.

\textbf{URI-file\_to\_features.py} are rolul de a procesa cele doua fisiere de date si pe baza trasaturilor din key\_features sa constituie un nou fisier ce contine pentru fiecare URL din cele doua fisiere tipul acestuia si trasaturile gasite in el, precum si frecventa lor.

\textbf{all\_URI\_features} este rezultatul rularii scriptului URI-file\_to\_features.py si contine pentru fiecare URL din cele doua fisiere de date, tipul acestuia si trasaturile gasite in el, precum si frecventa lor, acestea fiind folosite pentru antrenarea si testarea modelului de SVM.

\textbf{get\_TestSet\_TrainSet.py} are rolul de a 
imparti datele prezente in all\_URI\_features in doua seturi de proportie 70-30. Aceste doua seturi fiind folosite pentru antrenarea si testarea modelului.

\textbf{Train\_set} constituie 70\% din totalul de exemple acumulate pentru antrenarea modelului, doar acestea fiind de fapt folosite pentru antrenarea sa.

\textbf{Test\_set} constituie 30\% din exemplele acumulate, aceste date fiind folosite pentru testarea acuratetii modelului dupa antrenare.

\textbf{Train\_SQLi\_model.py} realizeaza obtinerea modelului de SVM folosit de sistem pentru prevenirea atacurilor SQL injection. Pentru antrenare modelului este folosit setul de date din fisierul Test\_set si algoritmi pusi la dispozitie de biblioteca open source libsvm \cite{libsvm}.

\textbf{SQLi.model} reprezinta rezultatul proiectului. Acesta este testat cu ajutorul setului de date din fisierul Test\_set si ulterior integrat in sistemul propus pentru a fi folosit pentru prevenirea atacurilo SQL injection.

\newpage

\begin{figure}[h]
	\centering
	\includegraphics[width=0.8\textwidth]{source_code.png}
	\caption{Interactiunea dintre interfata grafica si celelalte module}
	\label{fig:source_code}
\end{figure}
Figura ~\ref{fig:source_code} prezinta care sunt fisierele, specific fiecarui modul, care sunt accesate in mod direct de catre interfata sau indirect de alte fisiere, in timpul rularii si relatiile dintre acestea. \\


\textbf{User Interface} reprezinta intreaga componenta ce realizeaza interfata de utilizator cu toate fisierele necesare realizarii ei, incorporate in compozitia sa. Aceasta a fost realizata de un proiect dezvoltat in .Net implicand multe fisiere cu scopul realizarii elementelor grafice. Aceste elementu nu vor fi tratate in aceasta sectiune.

\textbf{Proxy.py} reprezinta fisierul ce incorporeaza, respectiv leaga, tot codul ce constituie partea de "back end" a proiectului. In acest script de python se realizeara instantierea elementului de reverse proxy cu parametri de adrese IP si porturi aferente, precum si furnizarea metodelor de detectie implementate, componentei de reverse proxy.

\textbf{Twisted library} este o biblioteca open source scrisa in Python, ce ofera suport pentru diferite protocoale(TCP, UDP, SSL/TLS). Biblioteca a fost folosita pentru partea de cod ce ofera implementarea unui reverse proxy. Mare parte din fisierele oferite de acesta biblioteca au fost folosite fara a fi suprascrisa sau a li se aduce modificari ulterioare, insa in vederea atingeri scopului propus, asupra unor fisiere sursa au fost aduse mici modificari(internet.tcp.py).

\textbf{internet.tcp.py} este scriptul din biblioteca twisted ce ofera suportul pentru protocolul TCP. Acest fisiser a fost modificat pentru introducerea detectiei impotriva utilizatorilor de Tor. Script-ul integreaza lista realizata de proiectul "Tor activity monitor" pentru verificarea adresei utilizatorilor ce doresc sa stabileasca o conexiune TCP.




\section{Algoritmi, metode si API-uri}

In acesta sectiune se urmareste descrierea codului sursa folosit la realizarea sistemului propus si explicarea amanuntita a codului/metodelor considerate mai relevante, precum si a principalelor api-uri utilizate in implementarea acestuia. Abordarea codului sursa se realizeaza conform sub-proiectelor prezentate in sectiunea anterioara(Tor activity monitor, AQLi SVM, Interfata utilizator).

\subsection{Tor activity monitor}

Conform figurii ~\ref{fig:tor_activity_monitor}, acest proiect este realizat din 5 fisiere, acestea avand o relatie liniara intre ele.

Fisierul ce incepe ciclul de executie, run\_Get\_new\_Tor\_ips.sh, este un fisier de bash ce ruleaza in bucla infinta. Fisierul trebuie sa ruleze pe un sistem ce este functional non-stop si cu acces nelimitat la internet. La ore fixe acesta (12 am si pm si 6 am si pm), acesta lanseaza in executie scriptul de python Get\_new\_Tor\_ips.py.

Fisierul principal din acest proiect il reprezinta Get\_new\_Tor\_ips.py. Acesta descarca pagina "Tor Network Status" \cite{tot_status} si proceseaza datele de pe acestea, introducand in fisierul IPs\_Tor\_ activity.txt informatii referitoare la adresele IP gasite pe pagina si uptime-ul lor din ultimele 6 ore. Procesarea adreselor IP si extragerea valorii lor de uptime se poate observa in urmatoarele doua secvente de cod.

\lstset{language=python,frame=single, showstringspaces=false}
\begin{lstlisting}
time_up = row.findAll('td')[4].contents[0]
ip = row.findAll('td', attrs={'class':'iT'})[0].findAll('a',
 attrs={'class':'who'})[0].contents[0]


\end{lstlisting}

Pentru prelucrarea continutului paginii, acesta a fost downloadat in memoria programului, iar codul HTML rezultant a fost prelucrat cu ajutorul bibliotecii de python open source, BeautifulSoup. Codul de mai sus reprezinta extragerea valori de timp(uptime) si adresa IP careia aceasta corespunde. Variabila "row" fiind un elemnet din obiectul iterabil rezultat din initializarea bibliotecii BeautifulSoup cu codul HTML al paginii.

\lstset{language=python,frame=single, showstringspaces=false}
\begin{lstlisting}
days = time_up.split()[1]
if days == 'd':
    hours = 6
else:
    hours = int(time_up.split()[0])
    if hours > 6:
        hours = 6

\end{lstlisting}

In secventa de mai sus de cod, este indentificata valoarea corecta de uptime din ultimele 6 ore. Pentru cazul in care valoare de uptime este sub forma de zile sau aceasta este mai mare de 6 ore, ea se seteaza pe 6 ore, intrucat nu ne intereseaza decat activitatea din ultimele 6 ore.
\begin{lstlisting}
from bs4 import BeautifulSoup
from urllib.request import urlopen

pagesource = urlopen(page)
soup = BeautifulSoup(pagesource.read())
table = soup.findAll('table', attrs={'class': 'displayTable'})
\end{lstlisting}

Rezultatele rularii fisierului Get\_new\_Tor\_ips.py sunt actualizate in IPs\_Tor\_activity.txt. Acest fisier este de fapt un json in care se realizeaza dump la noul dictionar obtinut de  Get\_new\_Tor\_ips.py. Dictionarul este constituit din adresa IP ca si cheie si o liste .Structura acestor liste este realizata dintr-o serie de numere intre 0 si 6 ce reprezinta timpul total de uptime corespunzator sfertului respectiv de zi. Dimensiunea acestei liste este fixata la 30(zile)*4(sferturi de zi), pe masura ce un element nou este adaugat, primul element din lista fiind scos.

Urmatorul pas este filtrarea tuturor adreselor IP ce au un uptime mai mare de 7 zile. acest lucru este realizat de scriptul Get\_ips\_blacklist.py, iar rezultatele sunt stocate in Tor\_ips.txt, o adresa IP pe linie.

\subsection{SQLi SVM}
Desfasurarea proiectului incepe de la cele doua fisiere SQLi\_URLs si Clean\_URLs. In aceste doua fisiere se afla URL-uri complete din categoria conforma cu numele fiecarui fisier. Aceste fisiere sunt procesate de scriptul URI\_processor.py care are rolul de a uniformiza datele in aceasi encodare. Urmatoarea bucata de cod prezinta secventa ce transforma valorile hexa dintr-un URI in caractere si modul in care erorile de encodare sunt tratate(sunt printate pentru a fi tratate manual de catre programator):

\lstset{language=python,frame=single, showstringspaces=false}
\begin{lstlisting}
for index, sub_uri in enumerate(uri.split('%')):
    if sub_uri:
        if index == 0:
            new_uri = sub_uri
            continue
        try:
            hex_val = bytearray.fromhex(sub_uri[:2]).decode()
        except UnicodeDecodeError:
            print(uri + ' --- ' + sub_uri[:2])
            return ''
        new_uri += hex_val + sub_uri[2:]
\end{lstlisting}
Intrucat intr-un URL caracterul '\%' nu poate sa apara decat daca acesta este encodat('\%25'- valoarea pentru encodarea caracterului '\%'), indentificarea tuturor caracterelor encodate a fost facuta prin indentificarea tuturor caracterelor de tipul '\%' in URI. In cazul in care un astfel de caracter este gasit, se incearca conversia urmatoarelor doua caractere(asteptandu-se, conform conventiei, sa fie cifre), din valoarea in hexa corespunzatoare unui anumit caracter in caracterul in sine.


In urma procesarii datelor, rezulta cele doua fisiere SQLi\_URIs si Clean\_URIs, acestea avand acelasi continut cele anterioare insa in aceasi encodare. In urma obtinerii acestor doua fisiere, a fost realizata completarea listei de trasaturi(lista initiala este constituita din toate cuvintele cheie a limbajului SQL) cu caracterele speciale intalnite in acest limbaj. Scriptul Get\_secial\_chars\_frequency.py are rolul de a determina frecventa de aparitie a fiecarui caracter special in cele doua seturi de date, iar pe baza unei observatii umane, a fost realizata determinarea caracterelor ce constituie trasaturi rentabile. Urmatoarea bucata de cod este din scriptul Get\_secial\_chars\_frequency.py si realizeaza numararea URI-urilor(pentru comparare) din setul de date si frecventa caracterelor speciale("dict" este un dictionar cu cheile fiind caracterele speciale utilizate in limbaj):
\lstset{language=python,frame=single, showstringspaces=false}
\begin{lstlisting}
with open(args.f, 'r') as fd:
    for lines in fd.readlines():
        lines_nr += 1
        line = lines.strip()
        for keys in dict:
            if keys in line:
                dict[keys] += 1
                
\end{lstlisting}

Pentru indentificarea caracterelor relevante din limbajul SQL in comparatie cu cele folosite intr-un URL obijnuit, s-a ales numararea frecventei de aparitie a acestora in URL-uri normale, dar si in URL-uri ce contin atacuri de SQL injection. Numararea se face alternativ, rezultatele fiind furnizate ca doua seturi de date separate, bucata anterioara de cod reprezentand procesul doar pentru una dintre cele doua categorii. Dictionarul referit in cod este alcatuit dintr-un dictionar ce are ca si chei valoarea caracterelor specialae cautate, iar ca valoarea acestea sunt initializate pe zero pentru a fi incrementate o data cu numarul de aparitii ale caracterelor cautate. Pentru uniformizarea setului de date, intrucat distributia acestor caractere nu este tot timpul uniforma, se tine cont si de numarul de linii(pe fiecare linie se afla un URI distinct) in care au fost indentificate frecventele lor de aparitie.

Dupa determinarea caracterelor relevante, acestea au fost completate manual in fisierul key\_features, fisier ce insumeaza toate trasaturile ce sunt folosite in antrenarea modelului(fiecare linie contine o trasatura, numarul liniei fiind si indicele de referinta a trasaturii).

Pentru obtinerea datelor intr-un format ce poate fi procesat de biblioteca libsvm \cite{libsvm} a fost necesara transformarea acestora intr-un anumit format:
\begin{lstlisting}
+1 23:1 37:4 103:2
-1 54:1 77:1
\end{lstlisting}
Convertirea URI-urilor in formatul de mai sus este realizata de scriptul URI-file\_to\_features.py. Formatul de mai sus este reprezentarea fiecarui URL in functie de tipul acestuia si indicele trasaturii gasite in interiorul sau precum si frecventa de aparite a acestora(tip trasatura:frecventa trasatura:frecventa). Urmatoarea bucata de cod este extrasa din fisierul URI-file\_to\_features.py si realizeaza conversia unui URI in echivalentul sau in trasaturi:
\begin{lstlisting}
for keys in keywords_aux:
 if keys in uri:
    if keywords_list.index(keys) > 184:
      keywords_aux[keys] = uri.count(keys)
      if not uri.count(keys) == 0:
        ok = True
      else:
        sub_uri = uri.split(keys)
        for index, ele in enumerate(sub_uri):
          if index + 1 < len(sub_uri):
            if ele and sub_uri[index + 1] and not ele[-1].
             isalpha() and not sub_uri[index+1][0].isalpha():
              keywords_aux[keys] += 1
              ok = True
          elif not ele and sub_uri[index - 1]:
              keywords_aux[keys] += 1
              ok = True
\end{lstlisting}

\subsection{Interfata utilizator}
\begin{figure}[h]
	\centering
	\includegraphics[width=0.6\textwidth]{ui_home.png}
	\caption{Meniul "Home" in interfata grafica}
	\label{fig:ui_home}
\end{figure}
Figura ~\ref{fig:ui_home} prezinta design-ul paginii de "Home" din interfata grafica pusa la dispozitie utilizatorilor sistemului. \\

\begin{figure}[h]
	\centering
	\includegraphics[width=0.8\textwidth]{ui_configure.png}
	\caption{Meniul "Configure" in interfata grafica}
	\label{fig:ui_configure}
\end{figure}
Figura ~\ref{fig:ui_configure} prezinta design-ul paginii de "Configure" din interfata grafica, in care utilizatorul sistemului poate sa seteze interfetele si porturile care vor fi tratate in timpul rularii. \\

\begin{figure}[h]
	\centering
	\includegraphics[width=0.8\textwidth]{ui_monitor.png}
	\caption{Meniul "Monitor" in interfata grafica}
	\label{fig:ui_monitor}
\end{figure}
Figura ~\ref{fig:ui_monitor} prezinta design-ul paginii de "Monitor" din interfata grafica, in care utilizatorul poate sa urmareasca activitatea sistemului in timpul rularii. \\

\begin{figure}[h]
	\centering
	\includegraphics[width=0.8\textwidth]{sqli_respons.png}
	\caption{Raspunsul pentru SQL injection URL}
	\label{fig:sqli_respons}
\end{figure}

Figura ~\ref{fig:sqli_respons} prezinta cum arata raspunsul primit de la server de catre un utilizator dupa trimiterea unui request catre server, cu intentia de a realiza un atatc de tipul SQL injection. \\

\begin{figure}[h]
	\centering
	\includegraphics[width=0.8\textwidth]{tor_respons.png}
	\caption{Principalele module ale sitemului propus}
	\label{fig:tor_respons}
\end{figure}
Figura ~\ref{fig:tor_respons} prezinta cum arata raspunsul primit de la server de catre un utilizator al retelei Tor ce intentioneaza sa se conecteze la acesta. \\


\section{Fluxul executie programului}
%
%
%
%Conține detalii de implementare: 
%\begin{itemize}
%  \item organizarea codului sursă, organizarea logică a codului (module, ierarhii de clase)
%  \item descrierea claselor, funcțiilor, API-urilor importante ale aplicației
%  \item descrierea la nivel de implementare a algoritmilor principali
%  \item descrierea părților mai dificile
%  \item alte detalii de implementare relevante, specifice fiecărei aplicații
%\end{itemize}
%
%Descrierea implementării trebuie să reflecte modul în care ea corespunde (se mapează) design-ului. 
%
%Nu se vor da detalii irelevante. Descrierea codului trebuie gândită ca un ghid de parcurgere a codului sursă de către cineva care vrea să continue proiectul vostru. 
%
%Exemplu de cod:
%\lstset{language=C,frame=single, showstringspaces=false}
%\begin{lstlisting}
%# include <stdio.h>
%  
%int main (int argc, char **argv)
%{
%  int i;
%    
%  for (i=0; i<argc; i++)
%    printf("argv[%d] = %s\n", i, argv[i]);
%    
%  return 0;
%}
%\end{lstlisting}

%\chapter{Tests and Results}
 \chapter{Teste și rezultate experimentale}
\label{cap:rezultate}

%Ponderea acestui capitol relativ la întreaga lucrare este de 5-10\%.
%
%Aici sunt prezentate metodele de validare a soluțiilor/sistemului descris în capitolele anterioare, scenariile de testare a corectitudinii funcționale, a utilizabilității, performanței etc.   
%
%Rezultatele testelor experimentale necesită, în general interpretări (dacă rezultatele obținute corespund așteptărilor, intuițiilor cititorului, de ce apar variații/excepții etc.) și comparații cu rezultatele altor metode similare. 
%
%Sistemele de testare și testele propriu-zise trebuie descrise detaliat astfel încât să poată fi reproduse și de alții care poate vor să-și compare soluțiile lor cu a voastră (eventual, codul testelor poate fi pus în anexe). Dacă se poate alegeți pentru evaluarea sistemului vostru benchmark-uri (pachete de testare) dedicate, astfel încât comparația cu alte sisteme să poată fi făcută mai ușor. În plus, astfel de teste sunt mult mai complete și mai realiste decât cele dezvoltate de voi. Oricum, încercați ca testele efectuate să nu fie triviale, ci să acopere scenarii cât mai reale, mai complexe și mai relevante ale funcționării sistemului vostru. 

%\section{Functional Tests}
 \section{Teste de funcționalitate}
 
În testarea sistemului s-a pus accentul pe testare celor două funcționalități de protecție împotrivă atacurilor de SQL injection și a utilizatorilor de Tor. 

Pentru testarea modelului folosit în prevenirea atacurilor de SQL injection s-a folosit setul de date de test specificat și în capitolul ~\ref{cap:implementare}. Acest set a fost obținut din setul inițial de date de antrenare, acesta fiind împărțit în 2 seturi separate cu proporția de 70\%(set antrenare) și respectiv 30\%(set testare). 
 
\begin{center}
	\begin{tabular}{||c c c c||} 
		\hline
		Tip set de testare  & Preziceri corecte & Dimensiune set & Acuratete(\%) \\ [0.5ex] 
		\hline\hline
		Clean \& infected & 187596 & 189278 & 99.11 \\ 
		\hline
		Clean & 5466 & 6258 & 87.34 \\
		\hline
		Infected & 182130 & 193020 & 99.51 \\
		\hline
	\end{tabular}
\end{center}

Pe baza tabelului de mai sus se pot observa diferențe majore de performanta între detecția pe setul "Clean" și pe "Infected". Aceste diferențe, respectiv scăderi de performanță în cazul setului Clean, se datorează dimensiunii mult mai mici a setului de Clean folosit și în antrenarea modelului. 

Pentru testarea eficienței sistemului de blocare a adreselor IP ale rețelei Tor, s-a realizat o referință între datele colectate de modulul de monitorizare a activității rețelei Tor. 

În prima diagrama este prezentată diferența procentuală dintre adresele IP cu un uptime mai mare de 7 zile în ultima lună și cele cu un uptime mai mic. În cea de a doua diagramă este evidențiată diferența dintre uptime-ul total ale acestor adrese IP din ultima lună. Printr-o analiză simplă în paralel a datelor din cele două diagrame, se poate observa că deși doar 20.31\% din IP-urile folosite de rețeaua Tor sunt blocate, din timpul total de uptime din ultima lună, 75.6\% aparține acestor adrese IP .

\begin{figure}
	\centering
	\includegraphics[width=0.7\textwidth]{test_case_1.png}
	\caption{ Raportul dintre numărul IP-urilor de pe Blacklist și Whitelist dintr-o lună }
	\label{fig:test_1}
\end{figure}

\begin{figure}
	\centering
	\includegraphics[width=0.7\textwidth]{test_case_2.png}
	\caption{ Raportul dintre uptime-ul IP-urile de pe Blacklist și Whitelist din aceeași lună }
	\label{fig:test_2}
\end{figure}


\newpage
%\section{Performance Tests}
 \section{Teste de performanță}

Sistemul propus a fost testat pe două configurații diferite pentru PC-ul gazdă: Intel Core i7-6600U CPU cu 4 nuclee și o frecvență de 2.6 GHz, cu 16 GB RAM DDR4 și i7-4790k CPU cu 4 nuclee și o frecvență maximă de 4.0 GHz, cu 16 GB RAM DDR4. Ambele sisteme furnizând mult mai multe resurse decât cele necesare unei funcționari optime. În cea ce privește resursele minime necesare, acestea trebuie să fie cele necesare rulării unui sistem de operare Windows 10: un procesor cu o frecvență mai mare de 1.0 GHz și o memorie mai mare sau egală cu 2 GB de RAM. În cea ce privește capacitățile sistemului de a suporta conexiuni exterioare, acesta nu a fost testat decât manual, prin intermediul unor mașini virtuale. 

%User Manual
%\chapter{User Manual}
 \chapter{Manual utilizator}

\label{cap:user-manual}
%
%Descrie pașii de instalare și rulare a aplicației. Dacă dezvoltarea aplicației s-a bazat sau a presupus instalarea și configurarea unei infrastructuri (complexe), descrieți detaliat pașii pe care i-ați urmat (referințele utilizate) și mai ales abaterile voite sau necesare de la documenațiile referite. Încercați ca cineva care vă continuă tema să nu mai fie nevoit să mai piardă timp inutil cu pregătirea mediului de lucru și să poată trece cât mai repede la abordarea temei proptriu-zise a proiectului. 
%
%Indincați, de asemenea, explicit versiunile aplicațiilor, bibliotecilor folosite și salvați o copie a acestora pe CD-ul atașat lucrării. E posibil ca aplicația voastră să nu mai funcționeze la fel pe alte versiuni și e bine de știut acest lucru și,  în același timp, e bine ca mediul descris de voi să poată fi reprodus ulterior. 
%
%Se întinde pe aproximativ 2-3 pagini. 

\section{Instalarea proiectului}

Pentru instalarea sistemului, atat timp cat toate dependintele acestuia sunt implinite, utilizatorul nu mai trebuie sa faca nimic. Toate fisierele necesare sistemului au fost impachetate in modulul principal, singurul pas ce poate fi executat de utilizator este crearea unui shortcut catre executabilul ce lanseaza in executie sistemul.
\section{Utilizare}

Dupa deschiderea aplicatiei, utilizatorul este intampinat de meniul de Home al acesteia. In urmatoarea fereastra se poate observa designul acestuia: 

\begin{figure}[h]
	\centering
	\includegraphics[width=0.6\textwidth]{ui_home.png}
	\caption{Meniul "Home" in interfata grafica}
	\label{fig:ui_home}
\end{figure}

In partea superioara a ferestrei se afla in permanenta numele ferestrei curente(in cazul de fata fereastra Home). In meniul prezent in stanga ferestrei sunt situate celelate doua ferestre disponibile utilizatorului: Configure si Monitor. Utilizatorul poate sa navighze intre aceste ferestre dand un click pe numele ferestrelor, respectiv pe pictograma in forma de casuta pentru meniul Home. \\

\begin{figure}[h]
	\centering
	\includegraphics[width=0.7\textwidth]{ui_configure.png}
	\caption{Meniul "Configure" in interfata grafica}
	\label{fig:ui_configure}
\end{figure}

In fereasra Configure, utilizatorul poate sa seteze parametrii de rulare a sistemului. Acestuia i se prezinta doua rubrici: Server Settings si Proxy Settings. In partea de Server Settings, utilizatorul seteaza datele server-ului: adresa IP a acestuia si portul aferent pe care acesta accepta conexiunui. Salavarea sau suprascrierea acestor date se realizeaza prin apasarea butonului "save" din rubrica respectiva. In partea de Proxy Settings, utilizatorul poate sa introduca mai multe adrese IP si un port, pe care sistemul sa accepte conexiuni si sa le redirectioneze catre server. Dupa setarea parametrilor de mai sus, sistemul se poate porni/opri prin apasarea butonului din dreapta jos(cu textul "start"/"stop" dupa caz).
\newpage
\begin{figure}[h]
	\centering
	\includegraphics[width=0.8\textwidth]{ui_monitor.png}
	\caption{Meniul "Monitor" in interfata grafica}
	\label{fig:ui_monitor}
\end{figure}

In fereastra de Monitor, utilizatorul poate sa urmareasca activitatea sistemului. In cazul in care sistemul detecteaza un eveniment, acesta este afisat in interfata grafica in aceasta fereastre(conform exemplului din imagine). Daca utilizatorul doreste stergerea evenimentelor antrioare, aceasta se poate realiza prin click dreapta pe eveniment.

%\chapter{Conclusions}
 \chapter{Conluzii}
\label{cap:concluzii}
\section{Privire de ansamblu asupra sistemului}

În urma eforturilor depuse, s-a realizat un sistem de prevenirea a intruziunilor care îndeplinește aproape în întregime obiectivele inițial propuse, mici inconveniente fiind la partea de performanță în detecție a sistemului. În cea ce privește performanța sistemului, în cazul detecției atacurilor de SQL injection(precum se poate vedea și în capitolul 7) procentul de fals pozitiv este mult mai mare decât se intenționa inițial(12.66\% în loc de 3-4\%), însă acest lucru datorându-se în principal setului mic de date de antrenare avute la dispoziție. 

Prin implementarea detecției atacurilor SQL injection folosind tehnici de machine learning, s-au dobândit cunoștințe valoroase și experiență ce pot fi folosite în viitoare proiecte. Întrucât în practică, abordarea acestui subiect a reprezentat un lucru nou, o mare parte din timpul investit în dezvoltarea sistemului a reprezentat documentarea și acumularea de noi cunoștințe și experiență pentru rezolvarea unor probleme cu tehnici de machine learning. 

\section{Dezvoltari ulterioare}

Datorită unei abordări modulare a dezvoltării sistemului, acestuia îi pot fi adăugate cu ușurință noi funcționalități. 

Sistemul prezintă două posibilități majore de dezvoltări ulterioare, acestea fiind aferente celor două tipuri de protecție implementată. În cazul protecției împotrivă adreselor IP malițioase, lista folosită momentan se poate extinde cu ușurință sau prin mici modificări în cod se pot adaugă noi liste. 

Protecția bazată pe detecția unui anumit conținut în URL poate fi cu ușurință extinsă, prin adăugarea de noi model de machine learning sau noi logici de detecție, ambele cazuri necesitând mici modificări în codul sursă. 

%Cuprinde:
%
%\begin{itemize}
% \item un rezumat al contribuțiilor aduse: ce s-a realizat, relativ la ce s-a propus, în ce constă experiența acumulată, care au fost punctele dificile atinse și rezolvată, recomandări pentru alții care abordează tema, la ce este bun ce s-a obținut etc.
% 
% \item a analiză critică a rezultatelor obținute: avantaje, dezavantaje, limitări
% 
% \item o descriere a posibilelor dezvoltări și îmbunătățiri ulterioare
%\end{itemize}
%
%Poate fi organizat pe secțiuni, dacă se dorește.
%
%Se întinde pe aproximativ 1-2 pagini. 








%\addcontentsline {toc}{chapter}{Bibliography}
\bibliographystyle{IEEEtran}
\bibliography{thesis}%same file name as for .bib

\appendix

\definecolor{DarkGreen}{HTML}{008000}
\definecolor{DarkBlue}{HTML}{000080}
\lstdefinestyle{PythonStyle}{
	language=Python,
	basicstyle=\ttfamily\scriptsize,
	columns=fullflexible,
	showstringspaces=false,
	keywordstyle=\color{DarkBlue},
	stringstyle=\color{DarkGreen},
	morekeywords={yield}
}

\chapter{Codul sursa(principalele fisiere)}
\section{Reverse proxy}

\begin{lstlisting}[style=PythonStyle]
class BadURL(Resource):
  def render(self, request):
    return ""

class HTTPSReverseProxyResource(proxy.ReverseProxyResource,                              object):
  def getChild(self, path, request):
    global m

    features = uri_convertor.uri_to_features(request.uri)
    x0, max_idx = gen_svm_nodearray(features)
    label = libsvm.svm_predict(m, x0)

    if int(label) == 1:
    print('blocked-->sqli:' + request.uri)
    return BadURL()

    child = super(HTTPSReverseProxyResource, self).getChild(path, request)
    return HTTPSReverseProxyResource(child.host, child.port, child.path, child.reactor)

def main(args):

  ap = argparse.ArgumentParser()
  ap.add_argument('-c', type=str, default='./server.crt')
  ap.add_argument('-k', type=str, default='./server.key')
  ap.add_argument('-m', type=str, default='./sqli.model')
  ap.add_argument('--server-ip', type=str,default='localhost')
  ap.add_argument('--server-port', type=int, default=8080)
  ap.add_argument('--listen-ip', type=str, default = ['192.168.58.1', 'localhost'])
  ap.add_argument('--listen-port', type=int, default=443)
  ns = ap.parse_args(args[1:])

  global m
  m = svm_load_model(ns.m)

  if type(ns.listen_ip) == str:
    ns.listen_ip = ast.literal_eval(ns.listen_ip)
  myProxy = HTTPSReverseProxyResource(ns.server_ip, ns.server_port, '')
  site = server.Site(myProxy)
  if ns.c:
    with open(ns.c, 'rb') as fp:
    ssl_cert = fp.read()
    if ns.k:
      with open(ns.k, 'rb') as fp:
      ssl_key = fp.read()
      certificate = ssl.PrivateCertificate.load(ssl_cert,     	 ssl.KeyPair.load(ssl_key, crypto.FILETYPE_PEM), 
      		     	 crypto.FILETYPE_PEM)
    else:
      certificate = ssl.PrivateCertificate.loadPEM(ssl_cert)
      for ele in ns.listen_ip:
        try:
          reactor.listenSSL(ns.listen_port, site,          	certificate.options(), interface=ele)
        except error.CannotListenError:
          print('Error: ' + ele + ' not a valid interface in this context')
          exit(0)
        except Exception as e:
          print('Error: ' + e.message)
          exit(0)
  else:
    for ele in ns.listen_ip:
      try:
        reactor.listenTCP(ns.listen_port, site, interface=ele)
      except error.CannotListenError:
        print('Error: ' + ele + ' not a valid interface in this context')
        exit(0)
      except Exception as e:
        print('Error: ' + e.message)
        exit(0)
  reactor.run()

if __name__ == '__main__':
  main(sys.argv)
\end{lstlisting}

\section{Convertor de URI in trasaturi pentru modelul SVM}
\begin{lstlisting}[style=PythonStyle]

def convert_uri(uri):
  new_uri = ''

    for index, sub_uri in enumerate(uri.split('%')):
      if sub_uri:
        if index == 0:
          new_uri = sub_uri
          continue
      try:
        hex_val = bytearray.fromhex(sub_uri[:2]).decode()
      except UnicodeDecodeError:
        hex_val = ''
      except ValueError:
        print(uri + ' --- ' + sub_uri[:2])
        return ''
      except Exception as e:
        print(str(e))
        print(uri + '---' + sub_uri[:2])
        exit(0)

      new_uri += hex_val + sub_uri[2:]
    new_uri = new_uri.upper()

return new_uri


def uri_to_features(uri):

  global keywords
  global keywords_list

  uri = convert_uri(uri)

  keywords_aux = keywords.copy()
  ok = False

  try:
    for keys in keywords_aux:
      if keys in uri:
        if keywords_list.index(keys) > 184:
          keywords_aux[keys] = uri.count(keys)
          if not uri.count(keys) == 0:
            ok = True
        else:
          sub_uri = uri.split(keys)
          for index, ele in enumerate(sub_uri):
            if index + 1 < len(sub_uri):
              if ele and sub_uri[index + 1]:
                if not ele[-1].isalpha() and not                 	 sub_uri[index+1][0].isalpha():
                  keywords_aux[keys] += 1
                  ok = True
            else:
              if not ele and sub_uri[index - 1]:
                keywords_aux[keys] += 1
                ok = True
  except:
    ok = False

  if ok:
    features = {}
    for keys in keywords_list:
      if not keywords_aux[keys] == 0:
        features[keywords_list.index(keys)+1] =	keywords_aux[keys]
    keywords_aux.clear()
    return features
  else:
    keywords_aux.clear()
    return {}


def main(argv):

  parser = argparse.ArgumentParser()
  parser.add_argument('-u', type=str, help='uri to convert')

  args = parser.parse_args(argv[1:])

  if not args.u:
    print('No given string(uri)')
    exit(0)

  uri = convert_uri(args.u)

  print(uri)


if __name__ == '__main__':
  main(sys.argv)
\end{lstlisting}
\section{Monitorizarea adreselor IP ale retelei Tor}
\begin{lstlisting}[style=PythonStyle]
import re
import os
import json
from urllib.request import urlopen
from bs4 import BeautifulSoup

regex = re.compile(
  r'^(?:http|ftp)s?://' # http:// or https://
  r'(?:(?:[A-Z0-9](?:[A-Z0-9-]{0,61}[A-Z0-9])?\.)+(?:[A-Z]{2,6}\.?|[A-Z0-9-]{2,}\.?)|' #domain...
  r'localhost|' #localhost...
  r'\d{1,3}\.\d{1,3}\.\d{1,3}\.\d{1,3})' # ...or ip
  r'(?::\d+)?' # optional port
  r'(?:/?|[/?]\S+)$', re.IGNORECASE)

all_ip = {}

def isValidUrl(url):
  if regex.match(url) is not None:
    return True
  return False
  

def sort_ip_list(ip_list):

  from IPy import IP
  ipl = [(IP(ip).int(), ip for ip in ip_list]
  ipl.sort()
  return [ip[1] for ip in ipl]

def init_dictionary():

  all_ip['0.0.0.0'] = []

def pars():

  global all_ip
  if not os.path.isfile('/home/avid/work/scenarii/bitbucket/craw_tor/ips_tor_activity'):
    init_dictionary()
  else:
  with open('/home/avid/work/scenarii/bitbucket/craw_tor/ips_tor_activity', 'r', encoding='utf8') as fd:
    all_ip = json.load(fd)

  page = 'https://torstatus.blutmagie.de/index.php?SR=Uptime&SO=Desc'

  pagesource = urlopen(page)
  s = pagesource.read()
  soup = BeautifulSoup(s)
  table = soup.findAll('table',attrs={'class':'displayTable'})
  rows = table[0].findAll('tr', attrs={'class': 'r'})

  current_tor_ips = {}

  for row in rows:

    try:
      time_up = row.findAll('td')[4].contents[0]
      ip = row.findAll('td', attrs={'class': 'iT'})[0].findAll('a', attrs={'class': 'who'})[0].contents[0]

      days = time_up.split()[1]
      if days == 'd':
        hours = 6
      else:
        hours = int(time_up.split()[0])
        if hours > 6:
          hours = 6
    except:
      continue
    current_tor_ips[ip] = hours

  for ips in all_ip:
    if ips == '0.0.0.0':
      continue
    if ips not in current_tor_ips:
      all_ip[ips].append(0)
      continue
    all_ip[ips].append(current_tor_ips[ips])
    del current_tor_ips[ips]

  for ips in current_tor_ips:
    all_ip[ips] = all_ip['0.0.0.0'].copy()
    all_ip[ips].append(current_tor_ips[ips])

  all_ip['0.0.0.0'].append(0)

  with open('/home/avid/work/scenarii/bitbucket/craw_tor/ips_tor_activity', 'w', encoding='utf8') as fd:
    json.dump(all_ip, fd)

pars()
\end{lstlisting}

\end{document}
