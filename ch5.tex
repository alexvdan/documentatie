%\chapter{Analysis and Design}
\chapter{Analiză și proiectare}
\label{cap:analiza-si-proiectare}


\section{Cerintele sistemului}




Sistemul trebui sa indeplineasca urmatoarele cerinte \textbf{functionale}:
\begin{enumerate}
	\item Sa realizere conexiunea la un server HTTP/HTTPS si sa redirectioneze traficul primit catre acesta.
	\item Sa intercepteze traficul venit pe o anumita interfata si port prestabilit.
	\item Sa prelucreze request-urile primite de la clineti intr-un format specific clasificatorului de SQL injection.
	\item Sa nu redirectioneze reqesturile clasificate ca si SQL injection.
	\item Sa blocheze conectarea clientilor ce folosesc ip-uri clasificate ca ip-uri de Tor.
	\item Sa permita utilizatorului sa editez si sa vizualizeze lista ip-urilor de Tor.
	\item Sa prezinte in interfata grafica toate interventiile rezlizate asupra traficului(blocari de conexiuni sau de request-uri).
	\item Sa permita utilizatorului sa configureze modul de operare al sistemului.
\end{enumerate}

Sistemul trebuie, de asemenea, să aibă următoarele caracteristici \textbf{non-funcționale}:
\begin{enumerate}
	\item Sa fie usor de instalat si de folosit pentru orice utilizator, oricat de neexperimentat.
	\item Sa poata intercepta traficul de pe orice/oricate interfete disponibile.
	\item Sa poata rula pe orice sistem de operare Windows cu Python2 instalat.
	\item Sa aiba o rata de blocare de 100\% a ip-urilor de pe lista neagra, iar
	in cazul detectiei de SQL injection sa nu aiba detectii false pozitive mai mari 2-3\%
	si o acuratete generala de peste 90\%
\end{enumerate}

%Acest capitol descrie design-ul proiectului și cuprinde, în general: 
%\begin{enumerate}
%  \item ilustrarea arhitecturii generale și detaliate a sistemului implementat, care să evidențieze modulele componente și relațiile dintre acestea
%  \item stările prin care trece sistemul în decursul funcționării sale (diagrame de stare)
%  \item modul de interacțiune dintre module și funcționalitatea acestora ilustrată prin diagrame de secvențe
%  \item descrierea algoritmilor/metodelor pe care se bazează funcționarea sistemului dezvoltat
%  \item descrierea organizării/structurii eventualelor baze de date folosite
%  \item justificarea alegerilor/deciziilor făcute și analiza critică a acestora (avantaje și dezavantaje), prin comparație cu alte alternative posibile
%\end{enumerate}
%
%Ca idee generală, design-ul trebuie să fie prezentat independent de o implementare anume, în general, și de cea a voastră, în particular. De asemenea, descrierea design-ului trebuie să conțină toate elementele și detaliile necesare, astfel încât altcineva decât voi să poate realiza o implementare a lui, fără a fi nevoit să ia decizii arhitecturale sau organizare (adică, de design) și să vă contacteze pentru a-și lămuri anumite aspecte neclare.
%
%Capitolul trebuie organizat pe secțiuni și subsecțiuni astfel descrierea să urmeze un cors logic și ușor de urmărit. 
%
%Ponderea acestui capitol relativ la întreaga lucrare este de 25-35\%.
%
%
%\section{Examples: lists, figures, tables, equations}
%
%Așa arată o listă de elemente nenumerotate:
%\begin{itemize}
%  \item element 1
%  \item element 2
%  \item \dots
%\end{itemize}
%
%
%Așa arată o listă de elemente numerotare:
%\begin{itemize}
%  \item element 1
%  \item element 2
%  \item \dots
%\end{itemize}
%
%
%Așa arată o listă în text: 
%\begin{inparaenum}[(\itshape 1 \upshape)]
%  \item element 1, 
%  \item element 2, 
%  \item \dots
%\end{inparaenum}
%
%\textbf{Atenție}: orice tabel, figura sau ecuație (formulă) trebuie referite \textit{explicit} în text explicit (de genul: în Figura X este ulustrat \dots, în Tabelul Y se poate vedea \dots), pentru că Latex le poate plasa chiar și pe altă pagină decât acolo unde vrem noi să ne referim la ele. Vedeți exemple de mai jos!
%
%Tabelul~\ref{table:example} ilustrează un exemplu de tabel. Un editor on-line de tabele poate fi găsit la \url{http://www.tablesgenerator.com/}. 
%
%\begin{table}[t]
%\centering                          % tabel centrat 
%\begin{tabular}{|c|c|c|c|}          % 4 coloane centrate 
%\hline\hline                        % linie orizontala dubla
%Case & Method\#1 & Method\#2 & Method\#3 \\ [0.5ex]   % inserare tabel
%%heading
%\hline                              % linie orizontal simpla
%1 & 50 & 837 & 970 \\               % corpul tabelului 
%2 & 47 & 877 & 230 \\
%3 & 31 & 25 & 415 \\[1ex]           % [1ex] adds vertical space
%\hline                              
%\end{tabular}
%\caption{Nonlinear Model Results}   % titlul tabelului
%\label{table:example}                % \label{table:nonlin} introduce eticheta folosita pentru referirea tabelului in text; referirea in text se va face cu \ref{table:nonlin}
%\end{table}
%
%În Figura~\ref{fig:exemplu} 
%
%\begin{figure}
%    \centering
%    \includegraphics[width=0.5\textwidth]{image}
%    \caption{Numele figurii}
%    \label{fig:exemplu}
%\end{figure}
%
%
%Formula~(\ref{eq:example}) arată modul de calcul al lui $\Delta$:
%\begin{equation} \label{eq:example}
%    \Delta =\sum_{i=1}^N w_i (x_i - \bar{x})^2 .
%\end{equation}
%
%
%Algoritmul~\ref{alg:example} este un exemplu de descriere pseudo-cod a unui algoritm, preluat de la \href{http://en.wikibooks.org/wiki/LaTeX/Algorithms#Typesetting_using_the_algorithm2e_package}{http://en.wikibooks.org/wiki/LaTeX}. El utilizează pachetul \textit{algorithm2e}. Alternativ, puteți utiliza pachetele \textit{algorithmic} sau \textit{program}. 
%
%\begin{algorithm}
% \KwData{this text}
% \KwResult{how to write algorithm with \LaTeX2e }
% initialization\;
% \While{not at end of this document}{
%  read current\;
%  \eIf{understand}{
%   go to next section\;
%   current section becomes this one\;
%   }{
%   go back to the beginning of current section\;
%  }
% }
% \caption{How to write algorithms}
% \label{alg:example}
%\end{algorithm}
